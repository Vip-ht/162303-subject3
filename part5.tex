\chapter{Kết luận}
\paragraph{}
Những nỗ lực phát triển ứng dụng Tic-Tac-Toe và so sánh chúng bằng Flutter
và React có thể hữu ích trong việc xác định ưu điểm và nhược điểm của các framework này
khi xử lý các ứng dụng web và đa nền tảng. Cả hai giải pháp đều có thể tạo ra các ứng dụng đầy đủ chức năng
và tiện dụng với giao diện đơn giản và hiệu quả, nhưng đều có những điểm mạnh chính
do kiến trúc và phương pháp tiếp cận của chúng.
React Native đã trở thành giải pháp phù hợp nhất
dành cho việc phát triển ứng dụng cho môi trường ưu tiên thiết bị di động vì nó cung cấp
hiệu năng gần với các ngôn ngữ Native và các tính năng UI tương đương với Android và iOS. Công cụ kết xuất 
dựa trên widget và việc sử dụng biên dịch trực tiếp sang mã native đã đảm bảo rằng nhiều
vấn đề về hiệu năng được giảm thiểu, bao gồm thời gian cập nhật trang, và các chuyển đổi trang
được thực hiện mượt mà và nhanh chóng góp phần cải thiện trải nghiệm chung. Nó cũng tích hợp tính năng 
hot-reload giúp công việc diễn ra nhanh hơn và chứng minh sự tiện lợi của việc kiểm tra bất kỳ
vấn đề nào và khắc phục trong thời gian ngắn.
\paragraph{}
Mặt khác, React đã chứng tỏ là một framework rất mạnh mẽ để xây dựng các ứng dụng web, sử dụng DOM ảo và 
quản lý trạng thái tốt để làm cho ứng dụng dễ sử dụng nhất có thể. Mặc dù tốc độ render hơi chậm so với 
Flutter vì nó sử dụng JavaScript và HTML, vốn là các ngôn ngữ web, React có những tính năng 
tuyệt vời về mặt thiết kế và hoạt ảnh động hơn. Cấu trúc module của nó cho phép phát triển gọn gàng hơn, và 
dễ quản lý hơn so với một hệ thống tích hợp, đặc biệt là khi việc tích hợp đa trình duyệt và web là quan trọng. 
Người dùng cho biết tính tương tác dưới dạng hiệu ứng di chuột và hoạt ảnh CSS trong React đã làm cho sự hiện 
diện của chúng, đặc biệt là khi thiết kế web, mang lại cảm giác tốt hơn so với khi không sử dụng, nhưng một 
số người vẫn thích Flutter hơn khi làm việc trên thiết bị di động của họ.
\paragraph{}
Đây là một dấu hiệu rõ ràng cho thấy khi lựa chọn một framework, người dùng cần xem xét bản chất của dự án, 
các nền tảng dự định, cũng như tốc độ mong đợi. Điểm mạnh của Flutter là đối với các nhà phát triển quan tâm 
nhiều đến hiệu năng của ứng dụng trên thiết bị di động và giao diện của ứng dụng trên các nền tảng, đây là một 
lựa chọn tốt. Mặt khác, React vẫn phù hợp với các dự án cần khả năng tương thích trên web và giao diện người dùng 
dựa trên dữ liệu. Java và .NET có nhiều điểm tương đồng, và người dùng phải quyết định dựa trên cơ sở dự án tùy 
thuộc vào các yêu cầu về hiệu năng, thời gian phát triển bằng ngôn ngữ đích và các nền tảng đích cần hỗ trợ. 
Có thể xem xét việc sử dụng kết hợp các phương pháp được mô tả hoặc các cải tiến hơn nữa trong tương lai để thu 
hẹp khoảng cách về hiệu năng của các ứng dụng web dựa trên các framework này, cũng như nghiên cứu khả năng tăng 
cường sử dụng chúng trong các tình huống phức tạp và mở rộng hơn. Nghiên cứu này bổ sung vào nguồn tài liệu hiện 
có về phát triển đa nền tảng, bao gồm những phát hiện thực tế sau đây có thể giúp các nhà phát triển lựa chọn giữa Flutter và React.
