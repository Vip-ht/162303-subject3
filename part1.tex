\chapter{Giới thiệu}

Ứng dụng di động hiện nay là một mảng quan trọng trong thế giới công nghệ, nhờ sự phổ biến của các thiết bị như điện thoại thông minh và máy tính bảng. Khi việc sử dụng ứng dụng chuyển dần từ trình duyệt sang ứng dụng di động ở hầu hết mọi lĩnh vực, nhu cầu về những công cụ phát triển ứng dụng thông minh, hiệu quả và có thể chạy trên nhiều nền tảng ngày càng tăng. Trước đây, phát triển ứng dụng di động thường chỉ nhắm vào từng nền tảng riêng lẻ, và nhà phát triển phải viết hai phiên bản khác nhau cho iOS và Android. Điều này tạo ra nhiều khó khăn: doanh nghiệp phải thuê lập trình viên cho từng nền tảng riêng, khiến thời gian phát triển, chi phí và công sức bảo trì đều tăng lên đáng kể.

Luận văn này nghiên cứu và so sánh ba bộ khung (framework) phát triển ứng dụng đa nền tảng nổi bật nhất hiện nay: React Native, Kotlin Multiplatform, và Flutter. Đây đều là những công nghệ hiện đại, được nhiều công ty sử dụng và đánh giá cao. Mỗi công cụ có ưu, nhược điểm riêng và những lựa chọn phù hợp với những loại dự án di động khác nhau. Để đạt mục tiêu trên, nghiên cứu này so sánh ba framework dựa trên độ phức tạp kiến trúc, hiệu năng, sự dễ dàng khi phát triển, mức độ hỗ trợ của hệ sinh thái, và các trường hợp sử dụng (use cases). Từ đó, đưa ra kết luận chung giúp nhà phát triển và doanh nghiệp lựa chọn framework phù hợp nhất cho ứng dụng của họ.

\textbf{React Native} Một dự án được phát triển bởi Meta cho phép viết ứng dụng trên cả hai môi trường Android và iOS bằng ngôn ngữ JavaScript và framework React. Nó cho phép tái sử dụng mã ở mức độ cực kỳ cao, đồng thời mang lại sự tiện lợi của JavaScript nhưng vẫn có cảm giác gần như hiệu năng native.

\textbf{Kotlin Multiplatform (KMP)} là một framework tương đối mới được ra mắt nhằm hỗ trợ phát triển ứng dụng đa nền tảng, đồng thời vẫn giữ khả năng phát triển ứng dụng Android bằng ngôn ngữ native Kotlin. Đồng thời, Kotlin Multiplatform theo đuổi mục tiêu tái sử dụng mã và hiệu năng cao, và đạt được điều này bằng cách chia sẻ phần mã ở tầng logic nghiệp vụ, trong khi vẫn cần phát triển giao diện (UI) riêng cho từng nền tảng.

\textbf{Flutter} là một bộ công cụ UI được tạo ra bởi Google cùng với ngôn ngữ Dart để xây dựng các ứng dụng được biên dịch native tuyệt đẹp trên Mobile, Web và Desktop. Nó đi kèm với một loạt widget dựng sẵn và một engine render xuất sắc, được tối ưu hóa mạnh và ưu tiên trải nghiệm thiết kế material đa nền tảng. Đó là lý do mỗi framework đều có những ưu điểm và nhược điểm đặc trưng riêng[1]. Vì vậy, khi hiểu được kết quả mà một framework cụ thể mang lại, người ta có thể đưa ra lựa chọn đúng cho từng dự án cụ thể. Trong nghiên cứu này, các framework này được đánh giá một cách nghiêm ngặt dựa trên các yếu tố bao gồm tốc độ phát triển, trải nghiệm người dùng, khả năng bảo trì và tương thích của mã nguồn, mức độ hỗ trợ từ cộng đồng, và khả năng tương thích với các thư viện khác.

Do đó, luận văn này nhằm mô tả và so sánh các công nghệ có trong mỗi framework và đánh giá các điều kiện để lựa chọn một framework thay vì framework khác trong bối cảnh một dự án cụ thể. Khi ngành phát triển ứng dụng di động tăng trưởng qua từng năm, việc chọn đúng framework là điều thiết yếu để tối đa hóa thành công, bất kể nhà phát triển là cá nhân, doanh nghiệp nhỏ hay tập đoàn lớn [2]. Vì vậy, nghiên cứu này được kỳ vọng sẽ hữu ích cả trong việc tiếp thu hiểu biết lý thuyết lẫn hỗ trợ quá trình đưa ra quyết định của các nhóm phát triển ứng dụng di động khi họ phải chọn một framework đa nền tảng tối ưu.

\section{Mục tiêu nghiên cứu}
Mục tiêu chính của luận văn này là so sánh một cách cụ thể mức độ phù hợp của ba framework phát triển ứng dụng di động phổ biến nhất hiện nay, React Native, Kotlin Multiplatform và Flutter, trong việc phát triển ứng dụng di động đa nền tảng. Nhằm giúp các nhà phát triển và tổ chức đưa ra quyết định đúng đắn về các framework ứng dụng web phù hợp dựa trên hiệu quả, hiệu năng, khả năng mở rộng, mức độ hỗ trợ từ cộng đồng và khả năng tích hợp, nghiên cứu này tiến hành phân tích so sánh các chỉ số của các framework.

Trong quá trình đó, luận văn hướng đến việc hiểu các kịch bản tốt nhất và tệ nhất mà mỗi framework phù hợp để tạo ra một môi trường công nghệ di động, đồng thời rút ra các hướng dẫn hữu ích nhằm cải thiện quy trình phát triển ứng dụng di động. Cuối cùng, mục tiêu là cung cấp cho các nhà phát triển thông tin cần thiết để họ có thể đưa ra quyết định đúng đắn khi lựa chọn framework phù hợp nhất với nhu cầu dự án của họ, dù đó là công cụ dựng prototype, một ứng dụng sẵn sàng cho sản xuất, hay một framework có thể duy trì lâu dài.

\section{Cấu trúc luận văn}
Vì vậy, luận văn này được tổ chức để đảm bảo người đọc có thể theo dõi một cách hệ thống bằng cách xác định các công cụ và vấn đề mà chúng giải quyết trước khi đi sâu vào việc giải thích nhiều hơn về React Native, Kotlin Multiplatform và Flutter, cũng như các khác biệt về hiệu năng trong từng tính năng của mỗi công cụ. Luận văn được chia thành năm chương; mỗi chương tiếp cận phát triển ứng dụng đa nền tảng từ một góc nhìn riêng và đưa ra bằng chứng về hiệu quả của những công nghệ này trong các dự án thương mại.

Chương thứ hai cung cấp bối cảnh tổng quát của phát triển ứng dụng di động với thông tin về phát triển đặc thù theo từng nền tảng và sự phát triển của mô hình đa nền tảng. Nó giải thích các khái niệm cơ bản của framework đa nền tảng và những công nghệ tạo nên React Native, Kotlin Multiplatform và Flutter. Chương này vì thế mang đến góc nhìn rộng hơn cần thiết để làm nền tảng cho việc phân tích ba framework.

Chương thứ ba bàn về React Native cùng kiến trúc, tính năng và vòng đời phát triển của nó. Những thông tin như vai trò của JavaScript và React trong xây dựng UI, lớp cầu nối giữa mã native và JavaScript, tối ưu hiệu năng, các thư viện và framework sẵn có… được trình bày chi tiết. Bên cạnh đó, chương này so sánh các ưu và nhược điểm của React Native liên quan đến hỗ trợ cộng đồng, mức độ trưởng thành của hệ sinh thái và khả năng tích hợp module native.

Chương thứ tư được dành cho Kotlin Multiplatform, với phần trình bày ngắn gọn về những điểm độc đáo trong quá trình phát triển của nền tảng này. Khác với các framework khác, Kotlin Multiplatform cho phép chia sẻ mã ở tầng logic trong khi giao diện vẫn phải phát triển riêng cho từng nền tảng. Chương này cũng xem xét cách khai thác các tính năng của framework, khả năng tích hợp với mã native và sự mở rộng của hệ sinh thái.

Ở chương thứ năm, Flutter được trình bày chi tiết bao gồm việc sử dụng Dart và kiến trúc mà nó áp dụng để hiển thị các thành phần UI phong phú nhờ hệ thống widget. Trọng tâm thảo luận xoay quanh hiệu năng vượt trội của ứng dụng, khả năng giữ giao diện nhất quán giữa các nền tảng, và việc Flutter biên dịch trực tiếp đến mức gọi các widget native, đây là một lợi thế lớn. Chương này cũng thảo luận về mức độ được sử dụng ngày càng tăng của Flutter, trải nghiệm của nhà phát triển, môi trường và các công cụ, thư viện đi kèm. Nó cũng nêu ra các hạn chế và khó khăn của framework trong phát triển ứng dụng di động nhằm giúp người đọc đưa ra quyết định sáng suốt.
