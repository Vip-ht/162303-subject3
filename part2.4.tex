\section{Flutter}
\paragraph{}
Google đã phát hành Flutter, một bộ công cụ giao diện người dùng tuyệt vời mà họ phát triển để xây dựng ứng dụng đa nền tảng bằng cách sử dụng một cơ sở mã duy nhất. Điều này làm cho kiến trúc dựa trên widget của nó thực sự nhất quán và mượt mà hơn trên Android, iOS, web và thậm chí cả các thiết bị máy tính để bàn. 
\paragraph{}
Tính năng hot reload là một trong những điểm mạnh chính mà Flutter mang lại cho các nhà phát triển của họ, vì nó cho phép họ nhìn thấy các thay đổi ngay lập tức, giúp việc phát triển trở nên nhanh chóng. Hơn nữa, Flutter cung cấp đồ họa hiệu suất cao và hoạt ảnh mượt mà nhờ vào công cụ render Skia. 
\paragraph{}
Flutter, với hệ sinh thái plugin ngày càng phát triển, cộng đồng hỗ trợ mạnh mẽ, cùng với sự hỗ trợ từ Google, tiếp tục được chú ý nhiều hơn như một framework tuyệt vời cho phát triển ứng dụng hiện đại.

\subsection{Giới thiệu về Flutter và Dart}
\paragraph{}
Cho đến nay, hoặc cho đến thời điểm tôi viết tài liệu này, trên mạng không có nhiều thông tin về Flutter và Dart. 
\paragraph{}
Flutter là một bộ công cụ phát triển giao diện người dùng được Google phát triển, mã nguồn mở và được ra mắt vào năm 2017. Nó cho phép các nhà phát triển phát triển và triển khai các ứng dụng được biên dịch thay thế cho di động và web cũng như các ứng dụng máy tính để bàn từ một cơ sở mã duy nhất. Lợi thế lớn nhất là tất cả các widget giao diện người dùng của Flutter tạo ra giao diện người dùng đồ họa bằng chính công cụ của chúng, gọi là Skia. 
\paragraph{}
Flutter sử dụng ngôn ngữ Dart, cũng do Google phát triển, đặc trưng bởi cú pháp gọn gàng, tốc độ cao, hỗ trợ cả AOT và JIT. AOT giúp tăng tốc độ khởi động ứng dụng, trong khi JIT hỗ trợ hot reload giúp cập nhật nhanh trong quá trình phát triển [21].

\begin{figure}[H]
    \centering
    \includegraphics[width=\linewidth]{Picture3.png}
    \caption{Những lợi ích của Flutter [21]}\label{fig:loi_ich_flutter}
\end{figure}

\paragraph{}
Flutter có những ưu điểm như hot reload, phát triển nhanh hơn, dễ sử dụng, v.v., như được minh họa trong Hình~\ref{fig:loi_ich_flutter}, do đó làm cho Flutter trở thành một lựa chọn tốt cho thiết kế giao diện người dùng đa nền tảng. (Shevtsiv \& Striuk, 2021)  
\paragraph{}
Flutter cung cấp một bộ widget gốc tốt có thể trừu tượng hóa cho đa nền tảng và với bố cục thích ứng, không phụ thuộc vào widget nền tảng. Điều này khiến nó rất phù hợp để tạo ra các thiết kế bố cục chính xác cho cả Android, iOS, cũng như các thiết bị khác. Nhờ Google img2img, Flutter và Dart đang được sử dụng ngày càng nhiều cho phát triển đa nền tảng với quá trình phát triển rất mượt mà và hiệu suất gần giống như native.

\subsection{Lợi ích của Flutter trong việc tạo giao diện người dùng}
\paragraph{}
Flutter được thiết kế đặc biệt để lập trình viên xây dựng giao diện đẹp và cuốn hút. Cách tiếp cận độc đáo của nó trong thiết kế giao diện người dùng mang lại một số lợi ích:

\paragraph{}
\textbf{Kiến trúc dựa trên Widget:} Flutter có kiến trúc dựa trên widget, nơi mọi người có thể tạo các bố cục khác nhau với sự trợ giúp của các công cụ phù hợp. Thú vị là, mỗi nút bạn nhấn và mỗi bố cục bạn thiết kế trong Flutter đều là một widget có thể được gọi chung và chỉnh sửa một cách tổng thể để cải thiện sự tương tác [12].

\begin{figure}[H]
    \centering
    \includegraphics[width=\linewidth]{Picture4.png}
    \caption{Kiến trúc dựa trên Widget của Flutter [18]}\label{fig:kien_truc_flutter}
\end{figure}

Hình~\ref{fig:kien_truc_flutter} này cho thấy Kiến trúc Dựa trên Widget của Flutter bao gồm ba lớp: Lớp Ứng dụng (logic mô hình và kinh doanh với Blocs), Lớp Miền (kho dữ liệu) chịu trách nhiệm về luồng dữ liệu, Lớp Dữ liệu (API bên ngoài, Firebase trong trường hợp này và Hive). 
\paragraph{}
\textbf{Hiển thị Pixel-Chính xác:} Flutter sử dụng động cơ render Skia tạo ra giao diện người dùng trực quan và chất lượng cao trên các thiết bị và nền tảng [24]. Các nhà thiết kế và lập trình viên có thể tạo ra kiểu giao diện và cảm giác mà họ muốn mà không bị ràng buộc bởi giao diện và cảm giác của một bộ widget GUI cụ thể.

\begin{figure}[H]
    \centering
    \includegraphics[width=\linewidth]{Picture5.png}
    \caption{Flutter Framework [18]}\label{fig:flutter_framework}
\end{figure}

Hình~\ref{fig:flutter_framework} cho thấy kiến trúc của framework Flutter, với cấu trúc theo lớp — framework Dart, engine render và các tích hợp theo nền tảng cụ thể. 
\paragraph{}
\textbf{Tính nhất quán đa nền tảng:} Flutter hứa hẹn sự thống nhất trên các nền tảng khác nhau như Android và iOS cho một giao diện người dùng tương tự và cơ hội để tránh việc tốn nhiều thời gian đảm bảo chúng luôn nhất quán [25].

\begin{figure}[H]
    \centering
    \includegraphics[width=\linewidth]{Picture6.png}
    \caption{Mô hình hoá Flutter đa nền tảng [10]}\label{fig:flutter_da_nen_tang}
\end{figure}
Như được thấy trong hình~\ref{fig:flutter_da_nen_tang}, Flutter cung cấp nhiều lợi thế cho phát triển đa nền tảng: đầu tiên, cơ sở mã nhanh, nhẹ, hỗ trợ hot reload, nhanh và nhiều tùy chọn tùy chỉnh phong phú. 
\paragraph{}
\textbf{Hoạt ảnh tùy chỉnh:} Flutter có một lớp hoạt ảnh giúp người dùng tạo ra nhiều hoạt ảnh và chuyển đổi khác nhau cho ứng dụng của họ, do đó, luồng trong ứng dụng được quản lý tốt. 
\paragraph{}
\textbf{Thiết kế đáp ứng:} Hệ thống bố cục linh hoạt của Flutter mở rộng khả năng phát triển giao diện người dùng mẫu, có thể đáp ứng linh hoạt với nhiều kích thước và mật độ màn hình khác nhau. Những tính năng này khiến Flutter trở thành nền tảng được ưa chuộng cho các nhà phát triển muốn mang đến những ứng dụng chất lượng cao, giàu giá trị thẩm mỹ, được ra mắt trong thời gian ngắn nhất có thể.

\subsection{Hỗ trợ cộng đồng và hệ sinh thái}
\paragraph{}
Flutter có một cộng đồng trẻ và rất năng động, được đảm bảo nhờ sự tham gia của Google. Điều này làm cho nó khá mạnh mẽ, khiến mọi người sử dụng nó như một framework đa nền tảng [30]. 
\paragraph{}
Cộng đồng nhà phát triển năng động: Nó cũng có một trong những hệ sinh thái nhà phát triển lớn nhất đối với bất kỳ framework đa nền tảng nào trên thị trường. Các cộng đồng này cung cấp thư viện plugin và tài nguyên từ những người đóng góp để hỗ trợ và làm phong phú việc phát triển [14]. Có hàng ngàn chủ đề và câu trả lời trên GitHub và Stack Overflow liên quan đến các vấn đề và giải pháp về Flutter.

\paragraph{}
\textbf{Thư viện Plugin phong phú:} Flutter có danh sách mười tám plugin tự phát triển và hàng ngàn gói khác, giúp dễ dàng thêm các chức năng như GPS, Thanh toán và truy cập Camera. Chúng được cung cấp bởi kho gói có tên là Pub.dev, nơi cung cấp đa dạng công cụ để đáp ứng các yêu cầu khác nhau của ứng dụng. 

\paragraph{}
\textbf{Hỗ trợ từ Google và cập nhật thường xuyên:} Vì Google hỗ trợ Flutter, các nhà phát triển sẽ nhận được các bản cập nhật cho framework này cũng như các tính năng mới và sự hỗ trợ liên tục [11]. 
\paragraph{}
\textbf{Tài liệu toàn diện:} Tài liệu do Flutter cung cấp dễ theo dõi và cung cấp mọi chi tiết mà một nhà phát triển cần để sử dụng các tính năng của Flutter [09]. 

\paragraph{}
\textbf{Sự chấp nhận bởi các nhà lãnh đạo ngành:} Các doanh nhân và nhà phát triển đã áp dụng Flutter; ví dụ, Alibaba, eBay và BMW đã áp dụng nó để chứng minh rằng nó có khả năng xử lý các ứng dụng quy mô lớn, sẵn sàng cho sản xuất. 

\paragraph{}
Cộng đồng Flutter là một môi trường phong phú với sự hỗ trợ liên tục, phù hợp với nhu cầu và kỳ vọng của các nhà phát triển và tổ chức trong phát triển ứng dụng đa nền tảng.
