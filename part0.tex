\begin{center}
    \fontsize{25pt}{24pt}\selectfont 
    \textbf{\textrm{Tóm tắt}}
\end{center}

\paragraph{}
 Với  nhu  cầu  ngày  càng  tăng  về  các  giải  pháp  đa  nền  tảng  cung  cấp  trải  nghiệm  giống  nhau  trên  cả
 nền  tảng  iOS  và  Android,  phát  triển  ứng  dụng  di  động  đã  thay  đổi  đáng  kể  trong  thời  gian  gần  đây.
   Ba  framework  phát  triển  di  động  nổi  tiếng — React  Native,  Kotlin  và  Flutter — 
 được  so  sánh  trong  luận  án  này.  React  Native  dựa  trên  JavaScript  đã  trở  nên  phổ  biến  vì
 hệ  sinh  thái  rộng  lớn  và  khả  năng  tái  sử  dụng  mã  đa  nền  tảng.  Được  tạo  ra  như  một  ngôn  ngữ  Android,  Kotlin  đã
 mở  rộng  sức  hấp  dẫn  của  mình  với  Kotlin  Multi-platform  Mobile  (KMM),  cung  cấp  cho  các  nhà  phát  triển  sự  mượt  mà
 khả  năng  tương  thích  với  Android  và  khả  năng  tối  ưu  hóa  mạnh  mẽ  dành  riêng  cho  từng  nền  tảng.  Google  đã  tạo  ra  Flutter  UI
 bộ  công  cụ,  nổi  bật  với  việc  sử  dụng  ngôn  ngữ  lập  trình  Dart  và  cung  cấp  một  phần  mềm,
 trải  nghiệm  thẩm  mỹ  tuyệt  vời  trên  iOS  và  Android.

\paragraph{}
 Nghiên  cứu  này  xem  xét  các  framework  này  từ  nhiều  góc  độ,  chẳng  hạn  như  năng  suất  của  nhà  phát  triển,  sự  dễ  dàng
 của  việc  tương  tác  với native API,  đo đạc  hiệu  năng  và  hỗ trợ  của  các công ty,
 cộng  đồng. Qua nghiên  cứu  này  các nhà phát triển có được  một  cái  nhìn  tổng  quan  về  những  ưu  và
  nhược  điểm  cũng  như  hiểu  biết  sâu  sắc  về  mức  độ  phù  hợp  của các framework  đối  với  các  loại  ứng dụng  di  động  khác  nhau.
Điều này được đem lại  thông qua  đánh  giá  các  đặc  điểm  bằng  cả  phương  pháp  định  tính  và  định  lượng.
  Khi  lựa  chọn  một  framework  phát  triển  di  động  đáp  ứng  nhu  cầu  của
 dự  án  cụ  thể,  các  nhà  phát  triển,  quản  lý  dự  án  và  các  tổ  chức  sẽ  được  hưởng  lợi  từ
 so  sánh  phân  tích.  Cuối  cùng,  bằng  cách  chỉ  ra  những  sự  đánh  đổi  và  các  ứng  dụng  tốt  nhất  cho
 React  Native,  Kotlin  và  Flutter,  nghiên  cứu  này  góp  phần  vào sự phát triển của công  nghệ ứng dụng di  động.

\tableofcontents