\chapter{Tổng quan tài liệu}
\paragraph{}
Trong thập kỷ vừa qua, bối cảnh phát triển ứng dụng di động đã thay đổi rất nhiều; các nhà phát triển và doanh nghiệp đều tìm kiếm những cách thức đạt được giải pháp tốt, hiệu quả và nhanh chóng cho ứng dụng của họ. Theo cách truyền thống, ứng dụng di động thường được phát triển dành riêng cho từng nền tảng, một bản chạy trên iOS và một bản khác chạy trên Android. Nhưng sự xuất hiện của các framework phát triển đa nền tảng đã đánh dấu một cuộc cách mạng trong ngành, đem đến cho nhà phát triển khả năng viết một bộ mã duy nhất nhưng triển khai trên nhiều nền tảng khác nhau. Nhờ đó, hiệu quả tăng lên, chi phí phát triển giảm xuống, và thời gian đưa ứng dụng ra thị trường cũng rút ngắn đáng kể.
\paragraph{}
Phần này  xem xét hạ tầng phát triển ứng dụng di động cả ở hướng đặc thù theo từng nền tảng lẫn hướng đa nền tảng, sự phát triển của chúng và ảnh hưởng đối với ngành công nghiệp. Nội dung đưa ra cái nhìn sơ lược về các framework đa nền tảng quan trọng như React Native, Kotlin Multiplatform và Flutter, trình bày các đặc điểm chính và những điểm mạnh riêng của từng công nghệ. Chương này cũng thảo luận các khía cạnh kỹ thuật quan trọng quyết định việc lựa chọn framework, bao gồm tốc độ phát triển, hiệu năng, trải nghiệm người dùng, khả năng bảo trì mã, mức độ hỗ trợ từ cộng đồng và khả năng tương thích với thư viện bên thứ ba.
\paragraph{}
Chương này đồng thời  tạo tiền đề cho phần phân tích so sánh ở các mục sau bằng cách thiết lập nền tảng hiểu biết vững chắc về những công nghệ này. Những hiểu biết đó giúp các nhà phát triển, quản lý dự án và tổ chức cân nhắc framework nào phù hợp cho việc phát triển ứng dụng di động.
\section{Tổng quan về các framework phát triển di động}
Các framework phát triển di động đã tiến bộ đáng kể; ngày nay, có nhiều framework hỗ trợ nhà phát triển thiết kế và xây dựng ứng dụng có thể chạy trên nhiều nền tảng khác nhau. Các framework này cung cấp giải pháp và tài nguyên để viết phần giao diện (front end) của ứng dụng, đồng thời cải thiện hoạt động của chương trình để mang đến trải nghiệm nhất quán.
\subsection{Phát triển di động – Hành trình tiến hóa}
Môi trường phát triển di động đã chuyển dịch từ mô hình phát triển native cho từng nền tảng riêng biệt sang các dạng môi trường đa nền tảng phổ quát hơn. Ban đầu, nhà phát triển phải xây dựng mã nguồn hoàn toàn riêng cho từng nền tảng như iOS và Android, khiến thời gian phát triển kéo dài và chi phí bảo trì cao. Nhu cầu cải thiện và đổi mới, tạo ra sự đòi hỏi về nền tảng hiệu quả hơn, đã dẫn đến sự ra đời của các framework đa nền tảng [1]. Những công cụ thế hệ đầu tiên như Xamarin cho phép khả năng chia sẻ mã giữa hai nền tảng nhưng thường gặp vấn đề về hiệu năng và tích hợp. Một số giải pháp mới hơn như React Native, Flutter và Kotlin Multiplatform Mobile (KMM) đã giải quyết được các vấn đề đó, mang lại hiệu năng gần với native hơn và đi kèm bộ công cụ phát triển tốt hơn. Những bước tiến này xuất hiện do nhu cầu rút ngắn chu kỳ phát triển, giảm chi phí và khả năng triển khai ứng dụng có đặc tính tương đồng trên nhiều nền tảng khác nhau.
\subsection{Nhu cầu về giải pháp đa nền tảng}
Sự tồn tại của nhiều loại thiết bị và hệ điều hành di động làm tăng nhu cầu có những giải pháp đa nền tảng trong toàn ngành. Việc xây dựng một ứng dụng native riêng cho mỗi nền tảng lớn vừa tốn thời gian vừa không được khuyến khích [3]. Các framework chạy được trên nhiều nền tảng giúp nhà phát triển dễ dàng viết mã chỉ một lần mà vẫn chạy được trên nhiều nền tảng, từ đó tiết kiệm thời gian và chi phí. Chúng cũng đảm bảo ứng dụng có hành vi và giao diện nhất quán trên các thiết bị khác nhau, từ đó nâng cao mức độ hài lòng của người dùng. Hơn nữa, phát triển đa nền tảng dễ quản lý hơn, đặc biệt khi cập nhật ứng dụng, vì một lần cập nhật có thể áp dụng cho tất cả nền tảng cùng lúc. Sự phức tạp tăng dần của ứng dụng và mức độ cạnh tranh ngày càng gay gắt trong ngành ứng dụng di động cũng thúc đẩy nhu cầu sử dụng mô hình phát triển đa nền tảng [2].

Do đó, những thay đổi đang diễn ra này giúp ngành công nghiệp xác định được những phương pháp hiệu quả hơn trong việc đạt được hiệu năng và sự hài lòng của người dùng từ các framework phát triển di động. Xu hướng chuyển sang phát triển đa nền tảng giải quyết các vấn đề do kiểu phát triển native gây ra, đồng thời cung cấp cho người dùng những giải pháp di động dễ triển khai, mạnh mẽ và hiệu quả trên nhiều nền tảng.
\section{React Native}
\paragraph{}
React Native được tạo ra bởi Facebook. Đây cũng là một framework mã nguồn mở, cho phép bất kỳ ai xây dựng ứng dụng di động bằng JavaScript và React nổi tiếng. Ra mắt vào năm 2015, nó giúp việc phát triển ứng dụng cho cả iOS và Android trở nên dễ dàng, vì được thiết kế để sử dụng một bộ mã duy nhất có thể hoạt động trên tất cả các nền tảng mà nó hỗ trợ. React Native sử dụng các thành phần native, cho phép xây dựng các ứng dụng mang lại trải nghiệm người dùng tương tự như những ứng dụng native khác.
\begin{figure}[H]
    \centering
    \includegraphics[width=\textwidth]{Picture1.png}
    \caption{Xếp hạng công cụ [1]}
    \label{fig:xep_hang_cong_cu}
\end{figure}

Bảng xếp hạng các framework, thư viện và công cụ được yêu thích nhất được thể hiện trong Hình~\ref{fig:xep_hang_cong_cu}. Điều này nhằm thu hút sự chú ý đến mức độ phổ biến của các công cụ phát triển đa nền tảng như Flutter và React Native.

\subsection{Bối cảnh và các tính năng cốt lõi}
React Native được phát triển như một giải pháp cho nhiệm vụ phức tạp là giữ cho hai codebase khác nhau được đồng bộ. Các tính năng cốt lõi gồm có:
\begin{itemize}
  \item \textbf{Phát triển đa nền tảng:} Cho phép các nhà phát triển tạo một mã nguồn cho nhiều ứng dụng chỉ bằng JavaScript.
  \item \textbf{Hot Reloading:} Cho phép nhà phát triển xem các cập nhật theo thời gian thực trong quá trình phát triển – một yếu tố giúp tăng hiệu quả [4].
  \item \textbf{Thành phần Native:} Tận dụng các widget native nên giao diện và cảm giác sử dụng giống ứng dụng native trên hệ điều hành tương ứng.
  \item \textbf{Hệ sinh thái phong phú:} Được hỗ trợ bởi một cộng đồng lớn và cung cấp nhiều thư viện cũng như phần mở rộng để tăng cường chức năng.
\end{itemize}

\subsection{Điểm mạnh và hạn chế}
Bảng dưới đây thể hiện điểm mạnh và hạn chế của React Native.

\renewcommand{\arraystretch}{1.3}

\begin{table}[H]
\centering
\begin{tabular}{|p{6cm}|p{6cm}|}
\hline
\multicolumn{1}{|c|}{\textbf{Điểm mạnh}}
&
\multicolumn{1}{c|}{\textbf{Hạn chế}} \\ \hline
\textbf{Tái sử dụng mã:} Tách biệt các phần chức năng nhưng đồng thời cũng tạo ra nhiều codebase hơn, dẫn đến tốn thời gian và chi phí. & \textbf{Hoạt ảnh phức tạp:} Làm việc với các thao tác và hoạt ảnh phức tạp gặp khó khăn do API [5]. \\ \hline
\textbf{Hiệu năng:} Sử dụng API native khi render giao diện để đạt hiệu quả tốt nhất, gần giống ứng dụng native. & \textbf{Phụ thuộc module native:} Một số chức năng có thể cần code native, vì vậy dự án có thể cần đến chuyên gia từng nền tảng. \\ \hline
\textbf{Hỗ trợ cộng đồng:} Cộng đồng rất năng động và lớn mạnh, nền tảng nhận nhiều cập nhật qua các năm cùng nhiều tài nguyên hỗ trợ. & \textbf{Hiệu năng giảm:} Dù vận hành mượt, ứng dụng có thể gặp suy giảm hiệu năng so với ứng dụng native hoàn chỉnh, đặc biệt với các tác vụ tính toán nặng [4]. \\ \hline
\end{tabular}
\caption{Điểm mạnh và hạn chế của React Native}
\end{table}
\subsection{Tích hợp Native API}
\paragraph{}
React Native cung cấp một hệ thống linh hoạt để tích hợp với Native API, cho phép các nhà phát triển:
\begin{itemize}[label=\scalebox{1}{$\bullet$}]
    \item Tương  tác  với  các  tính  năng  phần  cứng  của  thiết  bị  bao  gồm  camera,  GPS  và các cảm  biến  thông  qua 
các  mô-đun tích  hợp.
    \item Cho phép các nhà phát triển viết mã Native vượt ra ngoài các thành phần được cung cấp và tăng cường chức năng của ứng dụng.
    \item Framework này cung cấp kết nối giữa JavaScript và mã gốc để trao đổi dữ liệu và giao tiếp.
\end{itemize}
\paragraph{}
Khả năng tích hợp này cho phép các ứng dụng React Native tạo ra các ứng dụng
có một số chức năng dành riêng cho nền tảng trong khi phát triển một ứng dụng duy nhất
được xây dựng từ một mã nguồn và không phải viết lại phần lớn mã.

\section{Ứng dụng và tính phù hợp của các Framework}
\subsection{Các trường hợp sử dụng React Native}
\paragraph{}
Về cơ bản, React Native là một framework phát triển ứng dụng đa nền tảng,
tập trung chủ yếu vào tốc độ phát triển và khả năng chia sẻ các thành phần
mã nguồn quan trọng giữa iOS và Android. Ưu điểm so sánh của nó là dựa trên
Java script, giúp các nhà phát triển web muốn chuyển sang phát triển ứng dụng
di động dễ dàng hơn.
\paragraph{Các ứng dụng mạng xã hội:}
  Một  số  ứng  dụng  quy  mô  lớn  như  Facebook,
 Instagram  và  Discord  đã  sử  dụng  React  Native  để  tạo  ra  các  ứng  dụng  phức  tạp,  bao  gồm
 các  thành  phần  có  khả  năng  đồng  bộ  hóa  dữ  liệu  thời  gian  thực  mạnh  mẽ  cũng  như  các  thành  phần  động  của  ứng  dụng.
 Do  khả  năng  tái  sử  dụng  các  phần  UI  và  quản  lý  trạng  thái,  nó  phù  hợp  nhất
 cho  các  ứng  dụng  như  vậy.
\paragraph{Ứng dụng Thương mại điện tử và Thị trường:}
Một số lý do khiến các công ty như Shopify và
Walmart chọn React Native bao gồm: React Native hỗ trợ các thành phần giao diện người dùng (UI) có thể tái sử dụng và nội dung động,
thanh toán, xác thực và kiểm kê dễ dàng tích hợp với API.
\paragraph{Tạo bản mẫu và Phát triển MVP:}
Điều này đặc biệt hữu ích cho bất kỳ công ty khởi nghiệp hoặc doanh nghiệp nào muốn tạo bản mẫu 
nhanh và kiểm chứng ý tưởng của mình. Nhờ chỉ có một mã cơ sở, các nhà phát triển có thể thấy được 
lợi ích từ thời gian phát triển ngắn hơn và chi phí thấp hơn, từ đó có thể tạo ra các sản phẩm khả thi tối thiểu 
(Minimum Viable Product) trong thời gian ngắn hơn.
\paragraph{Ứng dụng truyền phát trực tuyến nội dung và phương tiện:}
Một số ứng dụng phương tiện được phát triển bằng React Native bao gồm Netflix và Bloomberg 
vì nền tảng này dễ sử dụng và xử lý việc hiển thị các tính năng video hoặc âm thanh.
\paragraph{}
Tuy nhiên, việc sử dụng cầu nối JavaScript trong React Native có nghĩa là nó có thể kém phù hợp hơn 
với các ứng dụng yêu cầu hiệu suất tính toán tối ưu hoặc tích hợp chi tiết với các chức năng gốc của nền tảng.
\subsection{Các trường hợp sử dụng Kotlin đa nền tảng - KMM tiềm năng}
\paragraph{}
Kotlin đa nền tảng di động (Kotlin Multiplatform Mobile - KMM) phù hợp nhất để sử dụng trong các tình huống cần có 
logic nghiệp vụ giống hệt nhau trên cả hai hệ điều hành nhưng vẫn đảm bảo giao diện người dùng gốc nhất quán.
Do đó, giải pháp này rất hiệu quả đối với các tổ chức ưu tiên hiệu suất gốc và khả năng áp dụng vào các nền tảng cụ thể.
\paragraph{Ứng dụng dành cho doanh nghiệp:}
Phù hợp nhất với các ứng dụng nghiệp vụ mở rộng cho iPhone cũng như tất cả các nghiệp vụ lớn khác, 
bởi vì logic nghiệp vụ chỉ cần viết một lần và giao diện người dùng (UI) có thể được thiết kế giống 
nhau bằng cách sử dụng các tiện ích dành cho cả iPhone và Android, 
đồng thời tận dụng những yếu tố đồ họa tốt nhất từ cả hai.
Ví dụ: các ứng dụng yêu cầu xử lý dữ liệu, làm việc hiệu quả hoặc các quy trình nghiệp vụ cụ thể.
\paragraph{Ứng dụng công nghệ tài chính và ngân hàng:}
Để được tích hợp vào quy trình phát triển, cùng với các ứng dụng khác, một ứng dụng tài chính có thể cần có hiệu suất cao, 
bảo mật tốt và được kết nối chặt chẽ với các yếu tố cụ thể của nền tảng, ví dụ như nhận dạng sinh trắc học hoặc lưu trữ được bảo vệ. 
Đối với lĩnh vực này, KMM phù hợp với hiệu suất ở cấp độ nền tảng và khả năng sử dụng trực tiếp Native API.
\paragraph{Ứng dụng có Logic Nghiệp vụ Phức tạp:}
Các ứng dụng có tính toán phức tạp, xử lý dữ liệu
hoặc triển khai kiến thức, kỹ năng hoặc mô hình cụ thể (chăm sóc sức khỏe, hậu cần)
tận dụng logic chung trong KMM. Điều này có nghĩa là các thuộc tính của một nền tảng sẽ phản ánh
các thuộc tính của nền tảng khác, đồng thời tránh được bất kỳ sự ảnh hưởng nào đến hiệu quả.
\paragraph{Tích hợp Dự án Cũ:}
KMM được thiết kế để triển khai theo từng bước, lý tưởng cho các doanh nghiệp có ứng dụng cũ muốn chuyển sang đa nền tảng. 
KMM có thể được triển khai dần dần bởi các nhà phát triển, chia sẻ logic mới cho các tính năng cải tiến mới, đồng thời duy trì mã ứng dụng gốc hiện có.
\paragraph{}
Điểm hiệu quả của KMM là khi cần hiệu suất và giao diện giống như native, nhưng lại không hiệu quả trong các ứng dụng có giao diện người dùng rất nặng, 
ưu tiên giao diện người dùng hoặc thường xuyên thay đổi, nơi mà việc phát triển đa nền tảng hoàn toàn là lý tưởng, chẳng hạn như trong React Native hoặc Flutter.
\subsection{Các trường hợp sử dụng Flutter}
\paragraph{}
Flutter được biết đến với khả năng triển khai thiết kế hấp dẫn về mặt thị giác, và
ứng dụng chạy nhanh hơn. Tùy thuộc vào từng dự án, chẳng hạn như những dự án mà thiết kế UI/UX đóng vai trò
rất quan trọng, kiến trúc dựa trên widget, kết hợp với một công cụ kết xuất khác nhau sẽ phù hợp nhất với
dự án đó.
\paragraph{Ứng dụng thiết kế chuyên sâu:}
Điều làm cho Flutter trở nên lý tưởng là nó được tối ưu hóa cao cho việc sử dụng các hiệu ứng động dành riêng cho khách hàng, 
chuyển tiếp riêng biệt và chuyển động của giao diện. Các ứng dụng như vậy bao gồm Google Ads và Reflectly, 
nơi thư viện widget Flutter giúp giảm độ khựng của giao diện người dùng cũng như tăng cường sức hấp dẫn của ứng dụng.
\paragraph{Khởi nghiệp và Ứng dụng Đa nền tảng: }
Một lợi thế nữa đi kèm với Flutter đa nền tảng là tiết kiệm thời gian và chi phí trong quá trình phát triển, 
thế nên đây là lựa chọn tốt hơn cho các công ty khởi nghiệp hoặc bất kỳ công ty nào có ngân sách eo hẹp. 
Nó cũng rất phù hợp để xây dựng các MJP chuyên biệt nhằm kiểm tra nhu cầu thị trường.
\paragraph{Ứng dụng Truyền thông và Giải trí:}
Bất kỳ ứng dụng nào cần được làm mới liên tục, tạo hoạt ảnh hoặc hoạt động trên nhiều nền tảng như phát trực tuyến video hoặc trò chơi đều cần công cụ kết xuất Flutter. 
Để minh họa điều này, chúng ta có ví dụ thực tế về việc Alibaba đã tận dụng thành công Flutter để xây dựng ứng dụng hoạt động trên nhiều nền tảng với hiệu suất tuyệt vời.
\paragraph{IoT và Hệ thống nhúng:}
Nhờ những cập nhật gần đây, bao gồm hỗ trợ Web, Desktop, và giờ đây là cả IoT, Flutter đang dần trở thành một nền tảng được ưa chuộng. Khả năng lập trình
trang web để chạy trên mọi nền tảng mà vẫn mang lại hiệu suất ấn tượng và tuân thủ
cùng một mô hình là một trong những lý do tại sao nền tảng này phù hợp với trường hợp sử dụng này.
\paragraph{}
Đó là lý do tại sao nếu dự án có thể được phát triển chỉ bằng các thành phần UI và yêu cầu chu kỳ triển khai ứng dụng nhanh chóng, Flutter sẽ hiệu quả hơn; 
trong khi KMM sẽ hiệu quả hơn nếu cần tích hợp sâu với các tính năng dành riêng cho nền tảng và tối ưu hóa hiệu suất dành riêng cho nền tảng.


\section{Flutter}
\paragraph{}
Google đã phát hành Flutter, một bộ công cụ giao diện người dùng tuyệt vời mà họ phát triển để xây dựng ứng dụng đa nền tảng bằng cách sử dụng một cơ sở mã duy nhất. Điều này làm cho kiến trúc dựa trên widget của nó thực sự nhất quán và mượt mà hơn trên Android, iOS, web và thậm chí cả các thiết bị máy tính để bàn. 
\paragraph{}
Tính năng hot reload là một trong những điểm mạnh chính mà Flutter mang lại cho các nhà phát triển của họ, vì nó cho phép họ nhìn thấy các thay đổi ngay lập tức, giúp việc phát triển trở nên nhanh chóng. Hơn nữa, Flutter cung cấp đồ họa hiệu suất cao và hoạt ảnh mượt mà nhờ vào công cụ render Skia. 
\paragraph{}
Flutter, với hệ sinh thái plugin ngày càng phát triển, cộng đồng hỗ trợ mạnh mẽ, cùng với sự hỗ trợ từ Google, tiếp tục được chú ý nhiều hơn như một framework tuyệt vời cho phát triển ứng dụng hiện đại.

\subsection{Giới thiệu về Flutter và Dart}
\paragraph{}
Cho đến nay, hoặc cho đến thời điểm tôi viết tài liệu này, trên mạng không có nhiều thông tin về Flutter và Dart. 
\paragraph{}
Flutter là một bộ công cụ phát triển giao diện người dùng được Google phát triển, mã nguồn mở và được ra mắt vào năm 2017. Nó cho phép các nhà phát triển phát triển và triển khai các ứng dụng được biên dịch thay thế cho di động và web cũng như các ứng dụng máy tính để bàn từ một cơ sở mã duy nhất. Lợi thế lớn nhất là tất cả các widget giao diện người dùng của Flutter tạo ra giao diện người dùng đồ họa bằng chính công cụ của chúng, gọi là Skia. 
\paragraph{}
Flutter sử dụng ngôn ngữ Dart, cũng do Google phát triển, đặc trưng bởi cú pháp gọn gàng, tốc độ cao, hỗ trợ cả AOT và JIT. AOT giúp tăng tốc độ khởi động ứng dụng, trong khi JIT hỗ trợ hot reload giúp cập nhật nhanh trong quá trình phát triển [21].

\begin{figure}[H]
    \centering
    \includegraphics[width=\linewidth]{Picture3.png}
    \caption{Những lợi ích của Flutter [21]}\label{fig:loi_ich_flutter}
\end{figure}

\paragraph{}
Flutter có những ưu điểm như hot reload, phát triển nhanh hơn, dễ sử dụng, v.v., như được minh họa trong Hình~\ref{fig:loi_ich_flutter}, do đó làm cho Flutter trở thành một lựa chọn tốt cho thiết kế giao diện người dùng đa nền tảng. (Shevtsiv \& Striuk, 2021)  
\paragraph{}
Flutter cung cấp một bộ widget gốc tốt có thể trừu tượng hóa cho đa nền tảng và với bố cục thích ứng, không phụ thuộc vào widget nền tảng. Điều này khiến nó rất phù hợp để tạo ra các thiết kế bố cục chính xác cho cả Android, iOS, cũng như các thiết bị khác. Nhờ Google img2img, Flutter và Dart đang được sử dụng ngày càng nhiều cho phát triển đa nền tảng với quá trình phát triển rất mượt mà và hiệu suất gần giống như native.

\subsection{Lợi ích của Flutter trong việc tạo giao diện người dùng}
\paragraph{}
Flutter được thiết kế đặc biệt để lập trình viên xây dựng giao diện đẹp và cuốn hút. Cách tiếp cận độc đáo của nó trong thiết kế giao diện người dùng mang lại một số lợi ích:

\paragraph{}
\textbf{Kiến trúc dựa trên Widget:} Flutter có kiến trúc dựa trên widget, nơi mọi người có thể tạo các bố cục khác nhau với sự trợ giúp của các công cụ phù hợp. Thú vị là, mỗi nút bạn nhấn và mỗi bố cục bạn thiết kế trong Flutter đều là một widget có thể được gọi chung và chỉnh sửa một cách tổng thể để cải thiện sự tương tác [12].

\begin{figure}[H]
    \centering
    \includegraphics[width=\linewidth]{Picture4.png}
    \caption{Kiến trúc dựa trên Widget của Flutter [18]}\label{fig:kien_truc_flutter}
\end{figure}

Hình~\ref{fig:kien_truc_flutter} này cho thấy Kiến trúc Dựa trên Widget của Flutter bao gồm ba lớp: Lớp Ứng dụng (logic mô hình và kinh doanh với Blocs), Lớp Miền (kho dữ liệu) chịu trách nhiệm về luồng dữ liệu, Lớp Dữ liệu (API bên ngoài, Firebase trong trường hợp này và Hive). 
\paragraph{}
\textbf{Hiển thị Pixel-Chính xác:} Flutter sử dụng động cơ render Skia tạo ra giao diện người dùng trực quan và chất lượng cao trên các thiết bị và nền tảng [24]. Các nhà thiết kế và lập trình viên có thể tạo ra kiểu giao diện và cảm giác mà họ muốn mà không bị ràng buộc bởi giao diện và cảm giác của một bộ widget GUI cụ thể.

\begin{figure}[H]
    \centering
    \includegraphics[width=\linewidth]{Picture5.png}
    \caption{Flutter Framework [18]}\label{fig:flutter_framework}
\end{figure}

Hình~\ref{fig:flutter_framework} cho thấy kiến trúc của framework Flutter, với cấu trúc theo lớp — framework Dart, engine render và các tích hợp theo nền tảng cụ thể. 
\paragraph{}
\textbf{Tính nhất quán đa nền tảng:} Flutter hứa hẹn sự thống nhất trên các nền tảng khác nhau như Android và iOS cho một giao diện người dùng tương tự và cơ hội để tránh việc tốn nhiều thời gian đảm bảo chúng luôn nhất quán [25].

\begin{figure}[H]
    \centering
    \includegraphics[width=\linewidth]{Picture6.png}
    \caption{Mô hình hoá Flutter đa nền tảng [10]}\label{fig:flutter_da_nen_tang}
\end{figure}
Như được thấy trong hình~\ref{fig:flutter_da_nen_tang}, Flutter cung cấp nhiều lợi thế cho phát triển đa nền tảng: đầu tiên, cơ sở mã nhanh, nhẹ, hỗ trợ hot reload, nhanh và nhiều tùy chọn tùy chỉnh phong phú. 
\paragraph{}
\textbf{Hoạt ảnh tùy chỉnh:} Flutter có một lớp hoạt ảnh giúp người dùng tạo ra nhiều hoạt ảnh và chuyển đổi khác nhau cho ứng dụng của họ, do đó, luồng trong ứng dụng được quản lý tốt. 
\paragraph{}
\textbf{Thiết kế đáp ứng:} Hệ thống bố cục linh hoạt của Flutter mở rộng khả năng phát triển giao diện người dùng mẫu, có thể đáp ứng linh hoạt với nhiều kích thước và mật độ màn hình khác nhau. Những tính năng này khiến Flutter trở thành nền tảng được ưa chuộng cho các nhà phát triển muốn mang đến những ứng dụng chất lượng cao, giàu giá trị thẩm mỹ, được ra mắt trong thời gian ngắn nhất có thể.

\subsection{Hỗ trợ cộng đồng và hệ sinh thái}
\paragraph{}
Flutter có một cộng đồng trẻ và rất năng động, được đảm bảo nhờ sự tham gia của Google. Điều này làm cho nó khá mạnh mẽ, khiến mọi người sử dụng nó như một framework đa nền tảng [30]. 
\paragraph{}
Cộng đồng nhà phát triển năng động: Nó cũng có một trong những hệ sinh thái nhà phát triển lớn nhất đối với bất kỳ framework đa nền tảng nào trên thị trường. Các cộng đồng này cung cấp thư viện plugin và tài nguyên từ những người đóng góp để hỗ trợ và làm phong phú việc phát triển [14]. Có hàng ngàn chủ đề và câu trả lời trên GitHub và Stack Overflow liên quan đến các vấn đề và giải pháp về Flutter.

\paragraph{}
\textbf{Thư viện Plugin phong phú:} Flutter có danh sách mười tám plugin tự phát triển và hàng ngàn gói khác, giúp dễ dàng thêm các chức năng như GPS, Thanh toán và truy cập Camera. Chúng được cung cấp bởi kho gói có tên là Pub.dev, nơi cung cấp đa dạng công cụ để đáp ứng các yêu cầu khác nhau của ứng dụng. 

\paragraph{}
\textbf{Hỗ trợ từ Google và cập nhật thường xuyên:} Vì Google hỗ trợ Flutter, các nhà phát triển sẽ nhận được các bản cập nhật cho framework này cũng như các tính năng mới và sự hỗ trợ liên tục [11]. 
\paragraph{}
\textbf{Tài liệu toàn diện:} Tài liệu do Flutter cung cấp dễ theo dõi và cung cấp mọi chi tiết mà một nhà phát triển cần để sử dụng các tính năng của Flutter [09]. 

\paragraph{}
\textbf{Sự chấp nhận bởi các nhà lãnh đạo ngành:} Các doanh nhân và nhà phát triển đã áp dụng Flutter; ví dụ, Alibaba, eBay và BMW đã áp dụng nó để chứng minh rằng nó có khả năng xử lý các ứng dụng quy mô lớn, sẵn sàng cho sản xuất. 

\paragraph{}
Cộng đồng Flutter là một môi trường phong phú với sự hỗ trợ liên tục, phù hợp với nhu cầu và kỳ vọng của các nhà phát triển và tổ chức trong phát triển ứng dụng đa nền tảng.

\section{Năng suất của lập trình viên}
\paragraph{}
RN, Flutter và KMM khác nhau về năng suất của các nhà phát triển. Một trong những lý do chính khiến React Native được đánh giá cao là nó có chu trình phát triển rất nhanh, nhờ khả năng hot reload, giúp các nhà phát triển dễ dàng thấy các thay đổi được áp dụng trên màn hình. Nó cũng có một tập hợp mạnh mẽ các thư viện bên thứ ba giúp tạo ứng dụng nhanh hơn để triển khai các tính năng liên quan trong một ứng dụng.

\begin{figure}[H]
    \centering
    \includegraphics[width=\linewidth]{Picture7.png}
    \caption{So sánh Flutter và React từ 2019–2021 [18]}\label{fig:react_flutter_trend}
\end{figure}

\paragraph{}
Mức độ phổ biến phổ biến của Flutter so với React Native từ năm 2019 đến 2021 được so sánh trong Hình~\ref{fig:react_flutter_trend}, nơi chúng ta thấy sự gia tăng về mức độ phổ biến của Flutter trong khi mức độ phổ biến của React Native vẫn ổn định. Dữ liệu cho thấy việc áp dụng Flutter ngày càng tăng trong phát triển đa nền tảng [18]. 

\paragraph{}
Tương tự như React Native, Flutter sử dụng tính năng hot reload trong khi framework này có thư viện widget phong phú và hơn thế nữa, giúp có thể tạo ra các thiết kế giao diện người dùng phức tạp. Nó được tài liệu hóa khá đầy đủ, và khả năng tương tác ấn tượng với các IDE như Android Studio và VS Code cũng cải thiện hiệu quả. Tuy nhiên, một điểm quan trọng có thể tốn thêm thời gian cho các nhà phát triển, đặc biệt là những người chưa quen với ngôn ngữ Dart của Flutter, là độ khó khi tiếp cận.

\paragraph{}
KMM chủ yếu giải quyết việc chia sẻ logic nghiệp vụ trong khi để phần giao diện người dùng vẫn giữ nguyên native trên cả Android và iOS. Vậy nên, nó chưa hoàn toàn thích ứng cho thiết kế giao diện đa nền tảng bởi vì lý do đó làm năng suất kém hơn nhiều so với của React Native và Flutter. Trong thực tế, các nhóm vẫn phải viết một mã giao diện người dùng hoàn toàn khác cho mỗi nền tảng, điều này mặc dù có thể giúp rút ngắn thời gian phát triển một chút nhưng lại khiến người dùng cuối cảm thấy rằng những gì họ đang sử dụng không được tương đồng giữa các nền tảng.

\paragraph{}
Tóm lại, có thể nói rằng, mặc dù React Native và Flutter được đánh giá cao về việc tạo ra các ứng dụng đa nền tảng, KMM lại áp dụng cách tiếp cận ngược lại và tập trung vào tính chất sát với ứng dụng native, bỏ qua vấn đề thời gian khi tạo ra các ứng dụng.

\subsection{Benchmark hiệu năng}
\paragraph{}
Về mặt hiệu năng, Flutter ấn tượng hơn React Native và, ở một số khía cạnh, có thể cạnh tranh với KMM. Render engine Skia trực tiếp phủ lên giao diện người dùng mà không cần gọi đến các thành phần native, animation mượt mà hơn và hiệu năng đa nền tảng được cải thiện. Kiến trúc này cho phép hiệu năng gần như native ngay cả với giao diện người dùng lớn và phức tạp.

\paragraph{}
Một nhược điểm lớn của React Native là nó sử dụng cầu nối JavaScript để mã JavaScript tương tác với các thành phần native, điều này có thể làm chậm ở một số khía cạnh như animation hoặc tính toán. Tuy nhiên, các cải tiến hiện tại như React Native Fabric được ra đời nhằm giảm thiểu điều này.

\paragraph{}
KMM cung cấp hiệu năng native thực sự vì nó được biên dịch từ logic nghiệp vụ chung của framework thành các tệp nhị phân riêng cho từng nền tảng: Android sử dụng JVM và iOS sử dụng LLVM. Vì KMM dựa vào các thành phần giao diện người dùng của nền tảng, nó có hiệu năng tốt nhất cho các ứng dụng có tần suất tương tác cao của người dùng và sử dụng tài nguyên nhiều.

\paragraph{}
Tóm lại, kết quả cho thấy Flutter cung cấp hiệu năng tốt hơn cho các ứng dụng phụ thuộc nhiều vào giao diện người dùng. Khi nói đến các ứng dụng có thể liên quan đến các tác vụ tính toán phức tạp hoặc yêu cầu tích hợp sâu với các thành phần native, KMM là lựa chọn tốt nhất, còn React Native ở mức trung bình và có thể xử lý phát triển ứng dụng đa nền tảng vừa phải mà vẫn duy trì hiệu năng tốt.

\subsection{Tương tác với Native API}
\paragraph{}
Cả ba framework đều cho phép bạn tương tác với các native API và mặc dù chúng có thể có sự khác biệt nhỏ về cách thực hiện, nhưng một số có thể dễ sử dụng hơn các framework khác. React Native sử dụng cầu nối JavaScript để gọi các native API khác, cho phép sử dụng các tính năng của thiết bị. Nhưng đối với các chức năng tùy chỉnh, các nhà phát triển có thể cần tạo các module native riêng biệt theo nền tảng bằng Java hoặc Swift, điều này hơi phức tạp.

\paragraph{}
Platform Channels giúp Flutter dễ dàng giao tiếp với mã native nhờ việc sử dụng Dart. Điều này cũng làm cho việc tích hợp các API hoặc chức năng native như GPS, camera và cảm biến trở nên gọn gàng và hiệu quả trong quá trình phát triển. Thêm vào đó, Flutter có nhiều plugin sẵn có giúp giảm thiểu nhu cầu ngôn ngữ native riêng biệt trong hầu hết các trường hợp.

\paragraph{}
Về khả năng mở rộng, KMM gần gũi với ngôn ngữ native hơn về việc truy cập giao diện API và cung cấp khả năng giao tiếp trực tiếp với các native API mà không qua các lớp cầu nối. Nó cho phép gọi các native API từ mã Kotlin dùng chung hoặc viết mã riêng cho từng nền tảng cho các mục đích nâng cao. Điều này làm cho KMM đặc biệt hiệu quả khi phát triển các ứng dụng cần phụ thuộc nhiều vào chức năng native nhưng vẫn có thể chia sẻ logic nghiệp vụ.

\paragraph{}
Nhìn chung, về đánh giá tích hợp với các native API, KMM mang lại sự dễ dàng hơn so với hai framework còn lại, trong khi Flutter và React Native đủ đáp ứng cho hầu hết nhu cầu với trải nghiệm có phần tốt hơn, nhưng không đáng kể, thông qua Platform Channels.

\subsection{Hỗ trợ từ cộng đồng và doanh nghiệp}
\paragraph{}
React Native, các thành phần và các dependency của nó đã được phát triển từ năm 2015 và nhận được sự hỗ trợ đáng kể trong cộng đồng nhà phát triển. Nó có một kho thư viện bên thứ ba lớn, nhiều hướng dẫn và diễn đàn như GitHub và Stack Overflow. Nhiều công ty, bao gồm cả Facebook và Instagram, những người đã tạo ra nó, cũng như Airbnb, đã triển khai sử dụng React Native trong sản phẩm.

\paragraph{}
Google đã hỗ trợ Flutter từ một nền tảng tương đối mới được ra đời vào năm 2017 phát triển nhanh chóng. Nó có cơ sở người dùng tốt và các công ty như Alibaba, BMW, eBay sử dụng và dựa vào nó. Hệ sinh thái vẫn đang mở rộng, và các plugin cùng công cụ được thu thập trong thư viện pub.dev.

\paragraph{}
Macro-media tương đối lâu đời hơn và chỉ nhắm đến những người cần chia sẻ mã mà không viết chương trình bằng các ngôn ngữ khác nhau của Macro-media. Họ có số lượng cộng đồng khá nhỏ nhưng hoạt động tích cực và dần tăng so với React Native và Flutter nhờ sự hỗ trợ của JetBrains. IDE chính thức của cả hai nền tảng, IntelliJ IDEA và Android Studio, cung cấp sự hỗ trợ mạnh mẽ cho ngôn ngữ, thêm vào đó, sự phổ biến của Kotlin trong cộng đồng phát triển Android cũng có lợi cho KMM.

\paragraph{}
Do đó, khi xét đến cộng đồng mạnh và khả năng truy cập thư viện, công cụ và hỗ trợ, các dự án nên được phát triển bằng React Native và Flutter. Trong khi đó, các tổ chức yêu cầu mô phỏng ứng dụng native mạnh mẽ hơn và hiệu năng tối đa với hỗ trợ cộng đồng hạn chế nhưng ổn định, nên sử dụng KMM.

\input{part2.6.tex}