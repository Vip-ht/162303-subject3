\chapter{Phương pháp nghiên cứu}
\paragraph{}
Một nghiên cứu về năng suất phát triển được thực hiện theo nhiều khía cạnh: 
khả năng tương tác với API nền tảng, hiệu suất và sự hỗ trợ của cộng đồng
nhằm so sánh React Native, Kotlin và Flutter. Nghiên cứu này sử dụng kết hợp các
kỹ thuật định tính và định lượng để đưa ra đánh giá đầy đủ về các framework.
\paragraph{}
Trong nghiên cứu này, mục tiêu chính là đưa ra một phân tích toàn diện về hiệu suất,
năng suất của nhà phát triển và tương tác với API nền tảng, cũng  như  sự  hỗ  trợ  của  cộng  đồng  đối  với  React  Native,
 Kotlin  (Kotlin  đa  nền  tảng)  và  Flutter. Trong  thời  đại  ứng  dụng  di  động phát  triển,
 đặc  biệt  là  cho  các  ứng  dụng  đa  nền  tảng  cung  cấp  trải  nghiệm  giống nhau trong
 cả môi trường của  thiết  bị  iOS  và  Android,  các  framework  này  đều  rất  quan  trọng. Với  tính  năng  động
 bản  chất  của  hệ  sinh  thái di  động,  tất  cả  các  framework  này  cần  được  xem  xét
cẩn thận  để  hiểu  phương  pháp  nào  hiệu  quả,  tại  sao  và  áp  dụng  cho  loại  ứng  dụng  nào.
\paragraph{}
 Để  làm  được  điều  này,  nghiên  cứu sử dụng phương  pháp  tiếp  cận  hỗn  hợp.  Với  phương  pháp  này,  mỗi framework
 có  thể  được  hiểu  rõ  hơn  một  cách  toàn  diện  về  mặt  công  nghệ  cũng  như  kinh  nghiệm mà
 các  nhà  phát  triển nhận được từ  việc  sử  dụng  chúng.  Trong  quá  trình  đánh  giá,  các  ứng  dụng  được  tạo  ra  bằng  cách  sử  dụng  từng framework
 được  đánh  giá  về  mặt  hiệu  suất.  Các  ứng  dụng  được  đánh  giá  dựa  trên  thời  gian  khởi  động,  mức  sử  dụng  bộ  nhớ,
 Tiêu  thụ  CPU  và  khả  năng  phản  hồi  của  giao  diện  người  dùng.  Trong  các  ứng  dụng  này,  chúng  ta  có  thể  kiểm  soát
 môi  trường  để  đo  lường  và  so  sánh  các  số  liệu  cốt  lõi  về  hiệu  suất  cho  từng framework.  Ngoài  ra,
 chúng  tôi  đánh  giá  cách  các framework này  ảnh  hưởng  đến  tốc  độ  và  quá  trình  phát  triển  vì  chúng  đặt  ra
 những  thách  thức  liên  quan  đến  độ  phức  tạp  của  mã  và  thời  gian  xây  dựng/triển  khai.
\paragraph{}
Nghiên  cứu  này  nhằm  mục  đích  đưa  ra  đánh  giá  toàn  diện  và  cân  bằng  về  React  Native,
 Kotlin  và  Flutter.  Sau  khi  thu  thập  những  phát  hiện  này,  chúng  sẽ  có  giá  trị  đối  với  các  nhà  phát  triển,  quản  lý,
 và  các  tổ  chức  khác  để  xem framework nào  là  lựa  chọn  tốt  nhất  cho  dự  án  cụ  thể  của  họ
 yêu  cầu,  không  chỉ  xem  xét  hiệu  suất  kỹ  thuật  mà  còn  cả  sự  hài  lòng  của  nhà  phát  triển  và sự
 hỗ  trợ của cộng  đồng.  Cuối  cùng,  nghiên  cứu  này  có  thể  là  một  phần  của  cuộc  tranh  luận  đang  diễn  ra  về  đâu là các  phương  pháp  và  công  cụ tốt  nhất
   trong  phát  triển  ứng  dụng  di  động  trong  thế  giới  công  nghệ  thay  đổi  nhanh  chóng.
\section{Lựa chọn Framework}
\paragraph{}
 Ngành  công  nghiệp ứng dụng di  động  đã  dẫn  đến  việc  lựa  chọn  React  Native  cũng  như là Kotlin  (thông  qua
 Kotlin  Đa  nền  tảng  Di  động)  và  Flutter  vì  chúng  cung  cấp  các  giải  pháp đa  nền  tảng phù hợp với
 nhu  cầu  phát  triển.  Một  quá  trình  đánh  giá  kiểm  tra  từng  framework  bằng  cách  sử  dụng
 các  tiêu  chí  bao  gồm:
 \begin{itemize}[label=\scalebox{1.5}{$\bullet$}]
    \item Năng suất phát triển
    \item Khả năng tương tác với API nền tảng
    \item Hiệu năng
    \item Hỗ trợ từ cộng đồng
\end{itemize}
\paragraph{}
Đánh  giá  trong  nghiên  cứu  này  phụ  thuộc  cơ  bản  vào  các  framework đã chọn:  React Native
  cùng  với  Kotlin  (thông  qua  Kotlin  đa nền tảng)  và  Flutter  vì  chúng  xác  định
 cả  tính  hợp  lệ  và  tính  thực  tiễn  của đối  với nghiên  cứu phát  triển ứng  dụng  di  động  hiện  đại
 .  Nhiều  tổ  chức  khác  nhau  ở  cả  các  công  ty  khởi  nghiệp  nhỏ  và  các  doanh  nghiệp  lớn  đã  lựa  chọn
 những framework  này  vì  chúng  ngày  càng  được  các  nhà  phát  triển  ưa  chuộng.
 Các framework  đại  diện  cho  những  lựa  chọn  nổi  bật  vì  chúng  cung  cấp  các  tính  năng  khác  nhau  cùng  với
 các  mô  hình  lập  trình  đặc  biệt  và  môi  trường  nền  tảng  để  phân  tích.
\begin{figure}[h!]
    \centering
    \includegraphics[width=1\textwidth]{figure3.1.png}
    \caption{Các phần  tử  của  React  Native [23]}
    %\label{fig:example}
\end{figure}
\paragraph{}
 Framework  React  Native  đã  được  sử  dụng  trong  hình  3.1  bao  gồm  Redux  để  quản  lý  trạng  thái
 và  các  plugin  tùy  chỉnh  để  bổ  sung  thêm  chức  năng.
\paragraph{}
Framework  này  nhận  được  sự  ủng  hộ  to  lớn  trong  cộng  đồng,  đặc  biệt  là  từ  một  số  lượng  lớn
  nhà  phát  triển  đóng  góp  vào  việc  duy trì  framework này là mã  nguồn  mở.  React  Native  đã  được
 được  sử  dụng  để  trong  các  công  ty  lớn  như  Facebook,  Instagram  và  Airbnb, v.v.  
 đó là minh chứng cho sự linh  hoạt  và  mạnh  mẽ  trong thực tế.
\paragraph{}
 Đặc  biệt,  framework  này  tận  dụng  cầu  nối  để  giao  tiếp  giữa  mã native  và  JavaScript,
 và  trong  một  số  trường  hợp,  điều  này  có  thể  gây  ra  tình  trạng  nghẽn cổ chai  hiệu  suất.
\begin{figure}[h!]
    \centering
    \includegraphics[width=1\textwidth]{figure3.2.png}
    \caption{KMM và Flutter framework [25]}
    %\label{fig:example}
\end{figure}
\paragraph{}
 Framework KMM  được  hiển  thị  trong  hình  3.2,  cho  phép  chia  sẻ  logic  nghiệp vụ  giữa  iOS
 và  Android  trong  khi  vẫn  giữ  lớp  giao  diện  người  dùng  dành  riêng  cho  nền  tảng  trong  hình  3.2.  [25]
\paragraph{}
 Trong  số  những  lợi  ích  của  nó,  KMM  cực  kỳ  có  lợi  vì  nó  cho  phép tận  dụng các  tính  năng  ngôn  ngữ  hiện  đại,
súc  tích  và  an  toàn  về  kiểu  chữ của Kotlin trên  cả  hai  nền  tảng.  Bằng  cách  đó,  một  mã nguồn 
 cơ  sở chung  có  thể  được  viết  cho  logic  ứng  dụng,  mạng  và  truy  cập  cơ  sở  dữ  liệu  bởi
 các  nhà  phát  triển  và  UI  được  duy  trì  với  các  thành  phần  gốc  cho  từng  nền  tảng.  Tính  năng  tốt  nhất
 của  việc  sử  dụng  Kotlin  Đa  nền  tảng  so  với  các framework  khác là trừu  tượng  hóa  cả  UI  và  logic  thành  một
mã cơ  sở  duy nhất,  ở  đây  là  cách  tiếp  cận  này  cung  cấp  sự  cân  bằng  tuyệt  vời  giữa  hiệu  suất  gốc  và
 tái  sử  dụng  mã.
\paragraph{}
 Đặc  biệt,  các  nhóm  có  kinh  nghiệm  phát  triển  ứng  dụng  di  động bằng ngôn ngữ gốc (Java, Swift, v.v.)  được  thu  hút  
 vào  Kotlin, vì  điều  này  cho  phép  họ  chuyển  dần  sang  các  công  cụ  đa  nền  tảng .
\paragraph{}
 Do  đó,  các  nhà  phát  triển  không  quen  thuộc  với  Kotlin  hoặc  chỉ  đơn  giản  là  muốn  có  thứ  gì  đó  mới  lạ
 đối  với  phát triển  UI  có  thể  gặp phải  một  số  khó  khăn.
\section{Tiêu chí đánh giá}
\paragraph{}
Tiêu  chí  đánh  giá  cho  mỗi  framework  được  chia  thành  bốn  loại:
\subsection{ Năng  suất  của  nhà  phát  triển:}
\begin{itemize}[label=$\circ$]
    \item \textbf{Dễ học:}  Mức  độ  dễ  dàng  cho  các  nhà  phát  triển  (đặc  biệt  là  những  người  chưa  có  kinh  nghiệm)
  để  tìm  hiểu framework.
    \item \textbf{Công cụ hỗ trợ:} Có  sẵn  hỗ  trợ  IDE,  công  cụ  gỡ  lỗi,  thử  nghiệm
 framework  và  tích  hợp  với  quy  trình  CI/CD.
    \item \textbf{Khả  năng  tái  sử  dụng  mã:}  Số lượng  mã  có  thể  được  tái  sử  dụng  trên  các  nền  tảng  (iOS  và
 Android).
\end{itemize}
\subsection{Dễ  dàng  tương  tác  với  API  nền tảng:}
\begin{itemize}[label=$\circ$]
    \item \textbf{Truy  cập  vào  các  tính  năng  gốc:}   Dễ  dàng  truy  cập  và  tương  tác  với  các  API  nền tảng 
 như  máy  ảnh,  GPS  và  cảm  biến  thiết  bị.
    \item \textbf{Khả năng  tương  thích  đa  nền  tảng:} Mức  độ  trừu  tượng  hóa  các API đặc biệt nền  tảng
  trong  khi  vẫn  duy  trì  hiệu  suất  gốc.
\end{itemize}
\subsection{Hiệu  suất:}
\begin{itemize}[label=$\circ$]
    \item \textbf{Thời  gian  khởi  động:}   Thời  gian  cần  thiết  để  ứng  dụng  sẵn  sàng  cho  người  dùng
 sự  tương  tác.
    \item \textbf{Hiệu  suất  thời  gian  chạy:}   Độ  mượt  của  kết  xuất  UI,  bộ  nhớ  và  mức  sử  dụng  CPU
 trong  quá  trình  vận  hành  ứng  dụng.
    \item \textbf{Mức  tiêu  thụ  pin:}Ảnh  hưởng  đến  thời  lượng  pin  trong  quá  trình  sử  dụng  ứng  dụng  thông  thường.
\end{itemize}
\subsection{Hỗ  trợ của cộng  đồng  và  tổ chức:}
\begin{itemize}[label=$\circ$]
    \item \textbf{Quy mô  và  mức  độ  tham  gia  của  cộng  đồng: } Số  lượng  người  đóng  góp,  hoạt  động  trên  diễn  đàn,
 đóng  góp  nguồn  mở  và  sự  tham  gia  chung  vào  hệ  sinh  thái  nhà  phát  triển.
    \item \textbf{Tài  liệu  và  Hướng  dẫn:} Chất  lượng  và  độ  sâu  của  tài  liệu  chính  thức,
 cũng  như  sự  sẵn  có  của  các  hướng  dẫn  và  bài  giảng  từ  cộng  đồng.
    \item \textbf{Ứng  dụng  doanh  nghiệp:}  Ứng  dụng  từng  framework trong các  tổ  chức  lớn  và
 phù  hợp  với  các  ứng  dụng  doanh  nghiệp.
\end{itemize}
\section{Phương  pháp  phân  tích  dữ  liệu}
\paragraph{}
 Dữ  liệu  thu  thập  được  thông  qua  các đo đạc  hiệu  suất,  khảo  sát  nhà  phát  triển  và đánh giá từ cộng  đồng
  có  thể  được  phân  tích  bằng  cả  thống  kê  mô  tả  và  phân  tích  chuyên  đề.
\begin{itemize}
  \item 
\end{itemize}


