\section{React Native}
\paragraph{}
React Native được tạo ra bởi Facebook. Đây cũng là một framework mã nguồn mở, cho phép bất kỳ ai xây dựng ứng dụng di động bằng JavaScript và React nổi tiếng. Ra mắt vào năm 2015, nó giúp việc phát triển ứng dụng cho cả iOS và Android trở nên dễ dàng, vì được thiết kế để sử dụng một bộ mã duy nhất có thể hoạt động trên tất cả các nền tảng mà nó hỗ trợ. React Native sử dụng các thành phần native, cho phép xây dựng các ứng dụng mang lại trải nghiệm người dùng tương tự như những ứng dụng native khác.

Bảng xếp hạng các framework, thư viện và công cụ được yêu thích nhất được thể hiện trong Hình 1. Điều này nhằm thu hút sự chú ý đến mức độ phổ biến của các công cụ phát triển đa nền tảng như Flutter và React Native.

\subsection{Bối cảnh và các tính năng cốt lõi}
React Native được phát triển như một giải pháp cho nhiệm vụ phức tạp là giữ cho hai codebase khác nhau được đồng bộ. Các tính năng cốt lõi gồm có:
\begin{itemize}
  \item \textbf{Phát triển đa nền tảng:} Cho phép các nhà phát triển tạo một mã nguồn cho nhiều ứng dụng chỉ bằng JavaScript.
  \item \textbf{Hot Reloading:} Cho phép nhà phát triển xem các cập nhật theo thời gian thực trong quá trình phát triển – một yếu tố giúp tăng hiệu quả [4].
  \item \textbf{Thành phần Native:} Tận dụng các widget native nên giao diện và cảm giác sử dụng giống ứng dụng native trên hệ điều hành tương ứng.
  \item \textbf{Hệ sinh thái phong phú:} Được hỗ trợ bởi một cộng đồng lớn và cung cấp nhiều thư viện cũng như phần mở rộng để tăng cường chức năng.
\end{itemize}

\subsection{Điểm mạnh và hạn chế}
Bảng dưới đây thể hiện điểm mạnh và hạn chế của React Native.

\renewcommand{\arraystretch}{1.3}

\begin{table}[H]
\centering
\begin{tabular}{|p{6cm}|p{6cm}|}
\hline
\multicolumn{1}{|c|}{\textbf{Điểm mạnh}}
&
\multicolumn{1}{c|}{\textbf{Hạn chế}} \\ \hline
\textbf{Tái sử dụng mã:} Tách biệt các phần chức năng nhưng đồng thời cũng tạo ra nhiều codebase hơn, dẫn đến tốn thời gian và chi phí. & \textbf{Hoạt ảnh phức tạp:} Làm việc với các thao tác và hoạt ảnh phức tạp gặp khó khăn do API [5]. \\ \hline
\textbf{Hiệu năng:} Sử dụng API native khi render giao diện để đạt hiệu quả tốt nhất, gần giống ứng dụng native. & \textbf{Phụ thuộc module native:} Một số chức năng có thể cần code native, vì vậy dự án có thể cần đến chuyên gia từng nền tảng. \\ \hline
\textbf{Hỗ trợ cộng đồng:} Cộng đồng rất năng động và lớn mạnh, nền tảng nhận nhiều cập nhật qua các năm cùng nhiều tài nguyên hỗ trợ. & \textbf{Hiệu năng giảm:} Dù vận hành mượt, ứng dụng có thể gặp suy giảm hiệu năng so với ứng dụng native hoàn chỉnh, đặc biệt với các tác vụ tính toán nặng [4]. \\ \hline
\end{tabular}
\caption{Điểm mạnh và hạn chế của React Native}
\end{table}

