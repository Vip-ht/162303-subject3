\section{Tổng quan về các framework phát triển di động}
Các framework phát triển di động đã tiến bộ đáng kể; ngày nay, có nhiều framework hỗ trợ nhà phát triển thiết kế và xây dựng ứng dụng có thể chạy trên nhiều nền tảng khác nhau. Các framework này cung cấp giải pháp và tài nguyên để viết phần giao diện (front end) của ứng dụng, đồng thời cải thiện hoạt động của chương trình để mang đến trải nghiệm nhất quán.
\subsection{Phát triển di động – Hành trình tiến hóa}
Môi trường phát triển di động đã chuyển dịch từ mô hình phát triển native cho từng nền tảng riêng biệt sang các dạng môi trường đa nền tảng phổ quát hơn. Ban đầu, nhà phát triển phải xây dựng mã nguồn hoàn toàn riêng cho từng nền tảng như iOS và Android, khiến thời gian phát triển kéo dài và chi phí bảo trì cao. Nhu cầu cải thiện và đổi mới, tạo ra sự đòi hỏi về nền tảng hiệu quả hơn, đã dẫn đến sự ra đời của các framework đa nền tảng [1]. Những công cụ thế hệ đầu tiên như Xamarin cho phép khả năng chia sẻ mã giữa hai nền tảng nhưng thường gặp vấn đề về hiệu năng và tích hợp. Một số giải pháp mới hơn như React Native, Flutter và Kotlin Multiplatform Mobile (KMM) đã giải quyết được các vấn đề đó, mang lại hiệu năng gần với native hơn và đi kèm bộ công cụ phát triển tốt hơn. Những bước tiến này xuất hiện do nhu cầu rút ngắn chu kỳ phát triển, giảm chi phí và khả năng triển khai ứng dụng có đặc tính tương đồng trên nhiều nền tảng khác nhau.
\subsection{Nhu cầu về giải pháp đa nền tảng}
Sự tồn tại của nhiều loại thiết bị và hệ điều hành di động làm tăng nhu cầu có những giải pháp đa nền tảng trong toàn ngành. Việc xây dựng một ứng dụng native riêng cho mỗi nền tảng lớn vừa tốn thời gian vừa không được khuyến khích [3]. Các framework chạy được trên nhiều nền tảng giúp nhà phát triển dễ dàng viết mã chỉ một lần mà vẫn chạy được trên nhiều nền tảng, từ đó tiết kiệm thời gian và chi phí. Chúng cũng đảm bảo ứng dụng có hành vi và giao diện nhất quán trên các thiết bị khác nhau, từ đó nâng cao mức độ hài lòng của người dùng. Hơn nữa, phát triển đa nền tảng dễ quản lý hơn, đặc biệt khi cập nhật ứng dụng, vì một lần cập nhật có thể áp dụng cho tất cả nền tảng cùng lúc. Sự phức tạp tăng dần của ứng dụng và mức độ cạnh tranh ngày càng gay gắt trong ngành ứng dụng di động cũng thúc đẩy nhu cầu sử dụng mô hình phát triển đa nền tảng [2].

Do đó, những thay đổi đang diễn ra này giúp ngành công nghiệp xác định được những phương pháp hiệu quả hơn trong việc đạt được hiệu năng và sự hài lòng của người dùng từ các framework phát triển di động. Xu hướng chuyển sang phát triển đa nền tảng giải quyết các vấn đề do kiểu phát triển native gây ra, đồng thời cung cấp cho người dùng những giải pháp di động dễ triển khai, mạnh mẽ và hiệu quả trên nhiều nền tảng.