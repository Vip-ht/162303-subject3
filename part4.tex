\chapter{Kết quả và thảo luận}
\paragraph{}
Trong chương này, tôi đã so sánh chi tiết giữa React Native, Kotlin Multiplatform và Flutter. Trong
phần so sánh này, chúng tôi đã thảo luận về sự khác biệt đáng kể về hiệu năng, tốc độ phát triển,
khả năng bảo trì mã nguồn, mức độ hỗ trợ kỹ thuật từ ecosystem, và khả năng tích hợp tốt như thế 
nào với các Native API.
Để có cái nhìn thực tế, chúng tôi đã bao gồm các ví dụ mã nguồn cho từng framework, minh họa cách
chúng xử lý các tác vụ phổ biến như UI rendering, quản lý trạng thái và tích hợp API. Phân tích này chỉ
ra điểm mạnh và điểm yếu của các framework, đồng thời giúp các developer và tổ chức hiểu rõ hơn xem framework
nào phù hợp với các yêu cầu của dự án họ.
\section{So sánh hiệu năng}
\paragraph{}
Trải nghiệm người dùng chắc chắn được cải thiện nhờ tốc độ hoạt động của ứng dụng di động, vì các ứng dụng di động đóng vai
trò thiết yếu đối với bất kỳ tác vụ hiệu năng nào. Trong phần dưới đây, chúng tôi đã phân tích cơ chế kết xuất,
mức sử dụng bộ nhớ và tốc độ thực thi của React Native, Flutter và Kotlin Multiplatform.
\subsection{React Native}
\paragraph{}
React Native giao tiếp với các mô-đun gốc thông qua một cầu nối (bridge) bằng JavaScript.
Điều này cho phép chúng ta phát triển cho các nền tảng khác nhau nhưng lại 
nảy sinh các điểm nghẽn về hiệu năng, đặc biệt là khi hoạt ảnh hoặc việc tính toán quá nặng.
\paragraph{}
{\large \textbf{Rendering a List}}
\begin{minted}[frame=lines, framesep=2mm, baselinestretch=1.2, fontsize=\footnotesize, linenos]{javascript}
import React from 'react';
import { FlatList, Text, View } from 'react-native';

const data = Array.from({ length: 1000 }, (_, i) => ({ id: i, title: `Item ${i}` }));

const App = () => (
  <FlatList
    data={data}
    renderItem={({ item }) => <Text>{item.title}</Text>}
    keyExtractor={item => item.id.toString()}
  />
);

export default App;
\end{minted}
\paragraph{}
\textbf{Ưu điểm:}Dễ dàng triển khai và tận dụng hệ sinh thái JavaScript.
\paragraph{}
\textbf{Nhược điểm:}Hiệu năng bị giảm sút đối với các danh sách lớn do sự phụ thuộc vào cầu nối JavaScript-Native.
\subsection{Kotlin Multiplatform}
\paragraph{}
Nền tảng Kotlin cung cấp khả năng chia sẻ logic nghiệp vụ trên các platform
khác nhau và sử dụng các thành phần native UI của chúng. Hệ quả là, các tác
vụ đặc thù của platform hoạt động hiệu quả hơn.
\paragraph{}
{\large \textbf{Logic nghiệp vụ dùng chung}}
\begin{minted}[frame=lines, framesep=2mm, baselinestretch=1.2, fontsize=\footnotesize, linenos]{kotlin}
    // Module được chia sẻ (dung chung cho Android và iOS)
expect class Platform() {
    val name: String
}

// Triển khai trên Android
actual class Platform actual constructor() {
    actual val name: String = "Android"
}

// Triển khai trên iOS
actual class Platform actual constructor() {
    actual val name: String = "iOS"
}
\end{minted}
\paragraph{}
\textbf{Ưu điểm: }Hiệu năng rất tốt đối với các tác vụ được tối ưu hóa cho
platform và tích hợp chặt chẽ với các Native API.
\paragraph{}
\textbf{Nhược điểm: }Yêu cầu việc triển khai UI riêng biệt cho từng platform.
\subsection{Flutter}
\paragraph{}
Trong Flutter, Dart là ngôn ngữ lập trình, và nó sử dụng Skia rendering engine
để tạo ra các UI có hiệu năng cao và khả năng tùy biến mạnh mẽ. Bản thân cầu nối
được loại bỏ trong quá trình biên dịch (Flutter không cần một cầu nối trung gian
lúc chạy như React Native, vì code Dart được biên dịch thẳng ra mã máy), vì vậy
nó trở thành mã gốc.
\paragraph{}
{\large \textbf{Rendering a List}}
\begin{minted}[frame=lines, framesep=2mm, baselinestretch=1.2, fontsize=\footnotesize, linenos]{dart}
import 'package:flutter/material.dart';

void main() {
  runApp(MyApp());
}

class MyApp extends StatelessWidget {
  final List<String> items = List.generate(1000, (i) => 'Item $i');

  @override
  Widget build(BuildContext context) {
    return MaterialApp(
      home: Scaffold(
        body: ListView.builder(
          itemCount: items.length,
          itemBuilder: (context, index) => ListTile(title: Text(items[index])),
        ),
      ),
    );
  }
}
\end{minted}
\paragraph{}
\textbf{Ưu điểm: }Hiệu năng xuất sắc, UI nhất quán trên các platform.
\paragraph{}
\textbf{Nhược điểm: }Dung lượng ứng dụng lớn hơn do phải bao gồm Flutter engine.

\section{Tốc độ phát triển}
Có các yếu tố như code reusability, hot reload, các prebuilt component và tính
sẵn có của chúng, quyết định tốc độ phát triển.

\subsection{React Native}
\paragraph{}
\textbf{Hot Reload: }Được hỗ trợ, cho phép các vòng lặp nhanh chóng.
\paragraph{}
\textbf{Code Reusability: }Cao, vì mã JavaScript có thể chạy trên cả iOS và Android.
\paragraph{}
\textbf{Ecosystem: }Hỗ trợ thư viện rộng lớn thông qua npm.
\paragraph{}
Nhưng một trong những tính năng chính nổi tiếng của React Native là khả năng
khởi động phát triển nhanh, bao gồm hot reload và code reusability. Kết hợp
với tất cả các tính năng này và một ecosystem rộng lớn mà nó hỗ trợ, việc
phát triển ứng dụng di động đa nền tảng thường là một lựa chọn hàng đầu.
\paragraph{}
\textbf{Hot Reload:}
\paragraph{}
Hot reload là một trong những tính năng phổ biến nhất
của React Native vì đội ngũ phát triển có thể thấy các thay đổi theo
thời gian thực mà không cần khởi động lại ứng dụng. Nó giúp cắt giảm đáng kể
thời gian của quy trình phát triển vì các lập trình viên có thể nhanh chóng thấy
kết quả của việc sửa đổi code ngay lập tức. Trong trường hợp của một nút bấm
hoặc những thay đổi về hành vi của một component, các thay đổi được áp dụng
trên trình giả lập hoặc thiết bị vật lý ngay lập tức. Bằng cách thực hiện quy trình
phát triển lặp lại, React Native giúp giảm thiểu thời gian chết và mở rộng năng
suất, biến nó thành một lựa chọn khả thi cho các team đang làm việc với
deadline gấp rút.
\paragraph{}
\textbf{Code Reusability:}
\paragraph{}
Với React Native, tái sử dụng mã nguồn là con át chủ bài trong phát triển ứng dụng vì
chúng cho phép các lập trình viên làm việc trên một codebase ứng dụng duy nhất
hoạt động trên cả nền tảng iOS và Android. Điều này được thực hiện bằng
cách sử dụng JavaScript và React framework, mang lại sự tương đồng cho các
chi tiết đặc thù của nền tảng khác nhau thông qua một API thống nhất. Các
component như nút bấm hoặc danh sách có thể được viết một lần và tái sử
dụng trên các nền tảng khác nhau mà không cần thay đổi nhiều. Điều này
không chỉ giảm thời gian phát triển mà còn giảm chi phí bảo trì cũng như
các bản cập nhật chỉ cần áp dụng một lần. Tuy nhiên, các lập trình viên cần viết
native module bằng Java, Objective-C hoặc Swift, điều này có thể mang lại
sự phức tạp.
\paragraph{}
\textbf{Ecosystem:}
\paragraph{}
Một lợi ích thực sự của React Native là nó hưởng lợi từ ecosystem
JavaScript rộng lớn – về cơ bản thông qua npm (Node Package Manager),
cung cấp cho bạn quyền truy cập vào hàng ngàn thư viện và công cụ.
Đây cũng là một ecosystem rộng lớn cho phép các lập trình viên sử dụng
các giải pháp được xây dựng sẵn cho các tác vụ phổ biến, trong đó có
quản lý trạng thái (Redux hoặc MobX), điều hướng (React Navigation)
và tích hợp API (Axios). Hơn nữa, việc tùy chỉnh React Native khá
dễ dàng, và cộng đồng xung quanh React Native cũng lớn và tích cực,
hỗ trợ tài liệu phong phú, hướng dẫn và nhiều tài nguyên bên ngoài.
Điều này giúp các lập trình viên dễ dàng tìm ra giải pháp cho các vấn đề
cũng như các phương pháp tốt nhất.
\begin{minted}[frame=lines, framesep=2mm, baselinestretch=1.2, fontsize=\footnotesize, linenos]{javascript}
import React, { useEffect, useState } from 'react';
import { View, Text, FlatList } from 'react-native';

const App = () => {
  const [data, setData] = useState([]);

  useEffect(() => {
    fetch('https://jsonplaceholder.typicode.com/posts')
      .then(response => response.json())
      .then(json => setData(json))
      .catch(error => console.error(error));
  }, []);
}
\end{minted}
\paragraph{}
\subsection{Kotlin Multiplatform}
\paragraph{}
\textbf{Hot Reload: }Mặc dù các công cụ như KMM (Kotlin Multiplatform Mobile) có
tốc độ phát triển nhanh hơn, nhưng nó không được hỗ trợ natively.
\paragraph{}
\textbf{Code Reusability: }Nó có các Logic nghiệp vụ dùng chung và UI code mang tính
đặc thù nền tảng.
\paragraph{}
\textbf{Ecosystem: }Đang phát triển, nhưng chưa hoàn thiện như React Native hay Flutter.
\paragraph{}
Kotlin Multiplatform (KMM) là một framework hiện đại được thiết kế để cung
cấp mã nguồn hiện đại dùng chung trên nhiều nền tảng mà không cần phải tái cấu trúc mã nguồn
chút nào; Nó cung cấp khả năng tái cấu trúc tự nhiên cho mã đặc thù nền tảng. Phương pháp này
mang lại lợi ích win-win vì nó có sự cân bằng phù hợp giữa khả năng tái sử dụng và tối ưu hóa
theo nền tảng, là một ứng viên sáng giá cho các project cần shared functionality nhưng lại cần
hiệu năng gốc và giao diện mang lại cảm giác quen thuộc cho người dùng.
\paragraph{}
\textbf{Hot Reload:}
\paragraph{}
Kotlin Multiplatform không hỗ trợ hot reload một cách tự nhiên, điều này
có nghĩa là các thay đổi chỉ có hiệu lực khi chạy ứng dụng 
(khác với React Native hay Flutter) và điều này có khả năng 
làm chậm quy trình phát triển, đặc biệt là khi làm việc với các UI 
component. Tuy nhiên, bằng cách sử dụng các công cụ như
Kotlin Multiplatform Mobile (KMM) và sự tích hợp với Android Studio,
việc tăng tốc độ phát triển trở nên dễ dàng hơn nhờ thời gian biên dịch ngắn hơn
và các công cụ gỡ lỗi tốt hơn. Mặc dù các công cụ này giúp giải quyết
một số khó khăn, nhưng tính năng hot reload thực thụ vẫn không có, và các
lập trình viên có thể phải tốn nhiều thời gian hơn để khởi động lại ứng dụng
nhằm xem các thay đổi, điều này kém hiệu quả hơn so với các framework
như Flutter hay React Native.
\paragraph{}
\textbf{Code Reusability:}
\paragraph{}
Kotlin Multiplatform vượt trội vì nó hỗ trợ việc chia sẻ các logic nghiệp vụ
giữa các nền tảng. Kotlin cho phép một phần mã nguồn của bạn
(như network, data processing, v.v.) được viết bằng Kotlin và
sử dụng trên cả Android, iOS và thậm chí là các ứng dụng web hoặc
desktop. Nó giúp bạn tránh việc sao chép và maintain cùng một
functionality trong phần lõi (core). Tuy nhiên, phần UI cần được
triển khai độc lập trên mỗi nền tảng vì Kotlin Multiplatform không
cung cấp một UI framework thống nhất. Điều này ngụ ý rằng các lập trình viên
cần viết UI code đặc thù cho từng nền tảng bằng Java/Kotlin cho Android và
Swift/Objective-C cho iOS, và việc này có thể làm tăng chi phí phát triển
đối với các UI phức tạp.
\paragraph{}
\textbf{Ecosystem:}
\paragraph{}
Tuy nhiên, ecosystem của Kotlin Multiplatform vẫn đang trong giai đoạn
phát triển tốt nhưng chưa hoàn toàn trưởng thành như React Native hay
Flutter. Bản thân Kotlin được hỗ trợ khá tốt, nhưng các công cụ và thư
viện của Kotlin Multiplatform vẫn đang trong quá trình hoàn thiện. Kotlin là
ngôn ngữ được tạo ra bởi JetBrains và họ đang tích cực đầu tư vào KMM,
cộng đồng cũng đang lớn mạnh với nhiều thư viện và plugin hơn được xây
dựng cho đa nền tảng. Tuy nhiên, các lập trình viên có thể vẫn tìm thấy
những khoảng trống trong ecosystem và sẽ cần xây dựng các giải pháp cụ
thể hoặc làm việc với thư viện được định nghĩa riêng cho nền tảng.
\paragraph{}
\subsection{Flutter}
\paragraph{}
\textbf{Hot Reload: }Cho phép thay đổi UI một cách nhanh chóng và dễ dàng.
\paragraph{}
\textbf{Code Reusability: }Cao, với một codebase duy nhất cho cả iOS và Android. 
(tạm thời chưa thể dịch, dịch bừa)
\paragraph{}
\textbf{Ecosystem: }Phát triển nhanh với nhiều package có sẵn.
\paragraph{}
Google là một trong những lý do khiến Flutter nhanh chóng trở nên phổ biến
như một framework đáng gờm để tạo ra các ứng dụng đa nền tảng, cho phép
bạn phát huy tối đa khả năng đồng thời đảm bảo ứng dụng vận hành mượt mà nhất.
Các thuộc tính nổi bật của nó, chẳng hạn như hot reload (văn bản gốc ghi là
hot extract), khả năng tái sử dụng code to lớn, và một ecosystem phong phú,
khiến nó trở thành một sự thay thế xứng đáng cho các lập trình viên có mong muốn
tạo ra các ứng dụng ổn định và tuyệt đẹp về mặt thẩm mỹ cho iOS và Android.
\paragraph{}
\textbf{Hot Reload:}
\paragraph{}
Khả năng Hot reload – một trong những tính năng được ca ngợi nhất của
Flutter – là khả năng nhìn thấy các thay đổi trong UI gần như ngay lập
tức mà không cần khởi động lại app. Tính năng này thực sự thúc đẩy quá
trình phát triển phần mềm một cách mạnh mẽ, đặc biệt là khi đang tinh chỉnh
giao diện người dùng hoặc gỡ lỗi. Ví dụ, nếu một lập trình viên thay
đổi padding của một widget hoặc thay đổi phối màu, nó sẽ phản ánh
trên màn hình ngay lập tức. Một phần quan trọng của quy trình làm
việc lặp lại (iterative workflow) này là tăng năng suất và duy trì
trải nghiệm lập trình viên tốt, khiến Flutter trở thành lựa chọn
lý tưởng cho việc tạo mẫu nhanh và cả các ứng dụng UI-heavy.
\paragraph{}
\textbf{Code Reusability:}
\paragraph{}
Flutter là một công cụ hữu ích mang lại khả năng tái sử dụng mã nguồn cao bằng
cách tạo ra một codebase duy nhất có thể sử dụng trên cả hai nền tảng
iOS và Android. Điều này được thực hiện thông qua Kiến trúc dựa trên Widget
của Flutter, cung cấp cho bạn số lượng khổng lồ các widget có khả năng tùy chỉnh
để xây dựng UI. Các widget được thiết kế để trông và hoạt động như các
thành phần bản địa trên mỗi nền tảng. Ví dụ, một nút bấm hoặc một dạng hiển thị danh sách
có thể được tạo một lần và sử dụng trên các nền tảng mà không cần
thay đổi, cũng như không cần bận tâm đến các điều chỉnh cho đặc trưng của mỗi nền tảng.
Với cách tiếp cận này, thời gian phát triển và công sức được giảm bớt, tính
bảo trì và khả năng mở rộng được đảm bảo, và kết quả là chúng ta tiết
kiệm được thời gian và công sức phát triển.
\paragraph{}
\textbf{Ecosystem:}
\paragraph{}
Flutter sở hữu một ecosystem phong phú với vô số các widget được dựng sẵn,
package và tool nhằm giúp việc phát triển phần mềm trở nên nhanh chóng hơn.
Không có gì mà Flutter không có package hỗ trợ, đặc biệt là đối với
quản lý trạng thái (như Provider hoặc Riverpod), điều hướng (như Flutter Navigator)
và tích hợp API (như Dio hoặc http package). Ecosystem của Flutter được
hậu thuẫn bởi Google và là một cộng đồng đang phát triển nhanh chóng,
điều này đồng nghĩa với sự cải tiến liên tục và hỗ trợ rộng rãi. Các lợi
thế khác của quá trình phát triển là bạn có thể dễ dàng tiếp cận các
tài liệu chi tiết, hướng dẫn, và cũng như các tài nguyên
do cộng đồng đóng góp.
\section{Code Maintainability}
\paragraph{}
Sự rõ ràng của codebase, tính môđun hoá và khả năng gỡ lỗi đơn giản
là những tham số quan trọng của khả năng bảo trì mã nguồn.
\subsection{React Native}
\paragraph{}
\textbf{Ưu điểm:} Vì JavaScript là một ngôn ngữ phổ biến, nên việc tiếp nhận những
cái mới các lập trình viên trở nên dễ dàng hơn.
\paragraph{}
Trước hết, React Native sử dụng JavaScript phổ biến, do đó hưởng lợi
từ sự quen thuộc rộng rãi với nó. Nó làm giảm nhu cầu tiếp nhận của các
lập trình viên mới (vì rất nhiều lập trình viên đã có kinh nghiệm phát
triển bằng JavaScript và React Framework). Tính mô-đun hóa thông qua các
component của React Native thúc đẩy kiến trúc dựa trên component cho
ứng dụng, nơi mà ứng dụng có thể được tách nhỏ thành các component có thể tái sử dụng.
Tính mô-đun hóa giúp việc bảo trì dễ dàng hơn vì sự thay đổi ở bất kỳ
component nào trong số này ít có khả năng ảnh hưởng đến những cái khác.
Hơn nữa, cơ sở khổng lồ các thư viện và công cụ thông qua npm giải quyết
các vấn đề điển hình về khả năng bảo trì như quản lý trạng thái và điều hướng.
\paragraph{}
\textbf{Nhược điểm:}Việc gỡ lỗi không dễ dàng vì nó liên quan đến một
JavaScript Native bridge (một cơ chế trung gian giúp mã JavaScript giao
tiếp được với các mô-đun gốc của hệ điều hành).
\paragraph{}
React Native gặp một số thách thức về khả năng bảo trì, khiến nó tuy là
một công cụ tuyệt vời nhưng vẫn tồn tại vài vấn đề về bảo trì.
JavaScript-Native bridge tạo thêm những điểm gây khó khăn (pain points)
khi thực hiện gỡ lỗi giữa JavaScript và mã gốc. Các lỗi nằm trong lớp
bridge đặc biệt khó chẩn đoán và sửa chữa. Hơn nữa, bản chất động của
JavaScript có thể gây ra các runtime error mà không bị phát hiện cho
đến khi thực thi thực tế, điều này không lý tưởng để đảm bảo độ tin
cậy của mã nguồn. Tuy nhiên, bằng cách sử dụng các công cụ như
TypeScript, một số vấn đề trên có thể được giảm thiểu.
Mặc dù vậy, chúng không được hỗ trợ một cách tự nhiên trong React Native,
đồng nghĩa với việc yêu cầu thêm nhiều công đoạn thiết lập và cấu hình.
\subsection{Kotlin Multiplatform}
\paragraph{}
\textbf{Ưu điểm:} Nó tương đối mới và hoạt động tốt với Android hơn
so với các ngôn ngữ khác.
\paragraph{}
Kotlin Multiplatform, dựa trên strong typing và null safety của
ngôn ngữ Kotlin, mang lại sự gia tăng đáng kể cho chất lượng mã nguồn
và khả năng bảo trì. Điều này làm cho codebase dễ dự đoán hơn và
dễ gỡ lỗi hơn vì các tính năng này giúp tìm ra lỗi tại compile time,
giảm thiểu nguy cơ xảy ra runtime crash.
Nó cũng làm tăng khả năng bảo trì vì các bản cập nhật hoặc sửa lỗi
chỉ cần được thực hiện một lần trong shared module và các cập nhật
đó sẽ tự động có sẵn trên tất cả các nền tảng. Cú pháp của Kotlin
cũng rất súc tích và các tính năng như data class và
extension function giúp mã nguồn sạch hơn và dễ đọc hơn.
\paragraph{}
\textbf{Nhược điểm:} Yêu cầu kiến thức về cả Kotlin và các ngôn ngữ
riêng biệt cho từng nền tảng. (Java/Swift).
\paragraph{}
Kotlin Multiplatform là một trong những thách thức lớn vì
Kotlin Multiplatform yêu cầu bạn viết UI code đặc thù cho
một một nền tảng. Mặc dù dùng chung các logic nghiệp vụ, mỗi nền tảng
phải triển khai UI layer riêng biệt và sử dụng ngôn ngữ gốc,
tức là Java/Kotlin cho Android, Swift/Objective-C cho iOS.
Điều này có thể làm cho codebase trở nên phức tạp hơn,
và nó có thể đòi hỏi kiến thức chuyên môn về nhiều ngôn ngữ
từ phía các lập trình viên. Thêm vào đó, ecosystem của
Kotlin Multiplatform vẫn đang hoàn thiện, nghĩa là các thư viện
và công cụ có thể bị thiếu và một số giải pháp thủ công sẽ trở
nên cần thiết, khiến việc bảo trì trở nên bất tiện.
\subsection{Flutter}
\paragraph{}
\textbf{Ưu điểm:} Khả năng định kiểu tốt của Dart và kiến trúc dựa
trên Widget của Flutter giúp tạo ra code sạch và thể hiện tính mô đun cao.
\paragraph{}
Nhờ được xây dựng trên ngôn ngữ Dart và lấy Widget làm trọng tâm,
Flutter giúp mã nguồn trở nên dễ bảo trì hơn. Cơ chế định kiểu tĩnh
(static typing) giúp cơ sở mã tăng độ tin cậy và giảm thiểu đáng kể
các lỗi thời gian chạy. Hơn nữa, kiến trúc dựa trên Widget cho phép
lập trình viên tổ chức ứng dụng như một cấu trúc cây phân cấp, bao
gồm các component độc lập có khả năng tái sử dụng. Cách tiếp cận
này giúp cải thiện tính mô-đun hóa, đồng thời làm cho mã nguồn
trở nên mạch lạc và dễ bảo trì hơn. Ngoài ra, tính năng hot reload
của Flutter cho phép các lập trình viên nhanh chóng iterate trên UI,
nhờ đó kiểm tra và tinh chỉnh các sửa đổi UI mà không cần phải khởi
động lại app. Hơn nữa, framework này cung cấp sẵn hàng loạt widget
và công cụ có thể dùng ngay lập tức. Điều này giúp hạn chế việc phải
tự viết code thủ công cho các tính năng, từ đó giảm bớt công sức bảo trì.
\paragraph{}
\textbf{Nhược điểm:} Một điểm trừ của Dart đó là còn khá xa lạ hơn đối với nhiều lập trình viên.
\paragraph{}
Một thách thức khác cần đối mặt khi dùng Flutter là mức độ phổ biến chưa cao,
chủ yếu do ngôn ngữ Dart chưa thực sự quen thuộc với phần lớn lập trình viên.
Thực tế thì Dart khá dễ học – đặc biệt đối với những lập trình viên đã có
kiến thức về lập trình hướng đối tượng. Tuy nhiên, Dart vẫn chưa thực sự
phổ biến rộng rãi khi so sánh với JavaScript hay Kotlin. Điều này có thể
khiến việc tìm kiếm và tiếp cận đối với các lập trình viên mới trở nên khó khăn,
hoặc việc tuyển dụng các lập trình viên Flutter giàu kinh nghiệm cũng không dễ dàng.
Hơn nữa, việc sử dụng rendering engine riêng của Flutter cũng khiến dung lượng ứng dụng
trở nên nặng hơn, đòi hỏi nhiều công sức tối ưu hóa để duy trì hiệu năng của ứng dụng
và trải nghiệm người dùng.
\section{Ecosystem và cộng đồng hỗ trợ}
\paragraph{}
Mức độ đóng góp của ecosystem và cộng đồng hỗ trợ hoàn toàn phụ thuộc vào việc
ecosystem đó lớn mạnh như thế nào.
\subsection{React Native}
\paragraph{}
Ecosystem: Trưởng thành, với số lượng lớn các thư viện và công cụ. Cộng đồng: Lớn mạnh và tích cực,
với tài liệu phong phú và các tài nguyên đến từ nhiều nguồn.
\subsection{Kotlin Multiplatform}
\paragraph{}
\textbf{Ecosystem:} Đang phát triển, với sự hỗ trợ mạnh mẽ từ JetBrains và cộng đồng Android.
Quy mô nhỏ hơn React Native và Flutter, nhưng đang tăng trưởng nhanh chóng.
\subsection{Flutter}
\paragraph{}
\textbf{Widgets và các package:} Rất đa dạng.
\paragraph{}
\textbf{Sự hậu thuẫn:} Lớn mạnh và tích cực từ Google.
