\chapter{Kết quả và thảo luận}
\paragraph{}
Trong chương này, tôi đã so sánh chi tiết giữa React Native, Kotlin Multiplatform và Flutter. Trong
phần so sánh này, chúng tôi đã thảo luận về sự khác biệt đáng kể về hiệu năng, tốc độ phát triển,
khả năng bảo trì mã nguồn, mức độ hỗ trợ kỹ thuật từ ecosystem, và khả năng tích hợp tốt như thế 
nào với các Native API.
Để có cái nhìn thực tế, chúng tôi đã bao gồm các ví dụ mã nguồn cho từng framework, minh họa cách
chúng xử lý các tác vụ phổ biến như UI rendering, quản lý trạng thái và tích hợp API. Phân tích này chỉ
ra điểm mạnh và điểm yếu của các framework, đồng thời giúp các developer và tổ chức hiểu rõ hơn xem framework
nào phù hợp với các yêu cầu của dự án họ.
\section{So sánh hiệu năng}
\paragraph{}
Trải nghiệm người dùng chắc chắn được cải thiện nhờ tốc độ hoạt động của ứng dụng di động, vì các ứng dụng di động đóng vai
trò thiết yếu đối với bất kỳ tác vụ hiệu năng nào. Trong phần dưới đây, chúng tôi đã phân tích cơ chế kết xuất,
mức sử dụng bộ nhớ và tốc độ thực thi của React Native, Flutter và Kotlin Multiplatform.
\subsection{React Native}
\paragraph{}
React Native giao tiếp với các mô-đun gốc thông qua một cầu nối (bridge) bằng JavaScript.
Điều này cho phép chúng ta phát triển cho các nền tảng khác nhau nhưng lại 
nảy sinh các điểm nghẽn về hiệu năng, đặc biệt là khi hoạt ảnh hoặc việc tính toán quá nặng.
\paragraph{}
Rendering a List
\begin{figure}[h!]
    \centering
    \includegraphics[width=0.8\textwidth]{figure9.png}
    \caption{Ứng dụng React Rendering List}
    %\label{fig:example}
\end{figure}
