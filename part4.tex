\chapter{Kết quả và thảo luận}
\paragraph{}
Trong chương này, tôi đã so sánh chi tiết giữa React Native, Kotlin Multiplatform và Flutter. Trong
phần so sánh này, chúng tôi đã thảo luận về sự khác biệt đáng kể về hiệu năng, tốc độ phát triển,
khả năng bảo trì mã nguồn, mức độ hỗ trợ kỹ thuật từ ecosystem, và khả năng tích hợp tốt như thế 
nào với các Native API.
Để có cái nhìn thực tế, chúng tôi đã bao gồm các ví dụ mã nguồn cho từng framework, minh họa cách
chúng xử lý các tác vụ phổ biến như UI rendering, quản lý trạng thái và tích hợp API. Phân tích này chỉ
ra điểm mạnh và điểm yếu của các framework, đồng thời giúp các developer và tổ chức hiểu rõ hơn xem framework
nào phù hợp với các yêu cầu của dự án họ.
\section{So sánh hiệu năng}
\paragraph{}
Trải nghiệm người dùng chắc chắn được cải thiện nhờ tốc độ hoạt động của ứng dụng di động, vì các ứng dụng di động đóng vai
trò thiết yếu đối với bất kỳ tác vụ hiệu năng nào. Trong phần dưới đây, chúng tôi đã phân tích cơ chế kết xuất,
mức sử dụng bộ nhớ và tốc độ thực thi của React Native, Flutter và Kotlin Multiplatform.
\subsection{React Native}
\paragraph{}
React Native giao tiếp với các mô-đun gốc thông qua một cầu nối (bridge) bằng JavaScript.
Điều này cho phép chúng ta phát triển cho các nền tảng khác nhau nhưng lại 
nảy sinh các điểm nghẽn về hiệu năng, đặc biệt là khi hoạt ảnh hoặc việc tính toán quá nặng.
\paragraph{}
{\large \textbf{Rendering a List}}
\begin{minted}[frame=lines, framesep=2mm, baselinestretch=1.2, fontsize=\footnotesize, linenos]{javascript}
import React from 'react';
import { FlatList, Text, View } from 'react-native';

const data = Array.from({ length: 1000 }, (_, i) => ({ id: i, title: `Item ${i}` }));

const App = () => (
  <FlatList
    data={data}
    renderItem={({ item }) => <Text>{item.title}</Text>}
    keyExtractor={item => item.id.toString()}
  />
);

export default App;
\end{minted}
\paragraph{}
\textbf{Ưu điểm:}Dễ dàng triển khai và tận dụng hệ sinh thái JavaScript.
\paragraph{}
\textbf{Nhược điểm:}Hiệu năng bị giảm sút đối với các danh sách lớn do sự phụ thuộc vào cầu nối JavaScript-Native.
\subsection{Kotlin Multiplatform}
\paragraph{}
Nền tảng Kotlin cung cấp khả năng chia sẻ logic nghiệp vụ trên các platform
khác nhau và sử dụng các thành phần native UI của chúng. Hệ quả là, các tác
vụ đặc thù của platform hoạt động hiệu quả hơn.
\paragraph{}
{\large \textbf{Logic nghiệp vụ dùng chung}}
\begin{minted}[frame=lines, framesep=2mm, baselinestretch=1.2, fontsize=\footnotesize, linenos]{kotlin}
    // Module được chia sẻ (dung chung cho Android và iOS)
expect class Platform() {
    val name: String
}

// Triển khai trên Android
actual class Platform actual constructor() {
    actual val name: String = "Android"
}

// Triển khai trên iOS
actual class Platform actual constructor() {
    actual val name: String = "iOS"
}
\end{minted}
\paragraph{}
\textbf{Ưu điểm: }Hiệu năng rất tốt đối với các tác vụ được tối ưu hóa cho
platform và tích hợp chặt chẽ với các Native API.
\paragraph{}
\textbf{Nhược điểm: }Yêu cầu việc triển khai UI riêng biệt cho từng platform.
\subsection{Flutter}
\paragraph{}
Trong Flutter, Dart là ngôn ngữ lập trình, và nó sử dụng Skia rendering engine
để tạo ra các UI có hiệu năng cao và khả năng tùy biến mạnh mẽ. Bản thân cầu nối
được loại bỏ trong quá trình biên dịch (Flutter không cần một cầu nối trung gian
lúc chạy như React Native, vì code Dart được biên dịch thẳng ra mã máy), vì vậy
nó trở thành mã gốc.
\paragraph{}
{\large \textbf{Rendering a List}}
\begin{minted}[frame=lines, framesep=2mm, baselinestretch=1.2, fontsize=\footnotesize, linenos]{dart}
import 'package:flutter/material.dart';

void main() {
  runApp(MyApp());
}

class MyApp extends StatelessWidget {
  final List<String> items = List.generate(1000, (i) => 'Item $i');

  @override
  Widget build(BuildContext context) {
    return MaterialApp(
      home: Scaffold(
        body: ListView.builder(
          itemCount: items.length,
          itemBuilder: (context, index) => ListTile(title: Text(items[index])),
        ),
      ),
    );
  }
}
\end{minted}
\paragraph{}
\textbf{Ưu điểm: }Hiệu năng xuất sắc, UI nhất quán trên các platform.
\paragraph{}
\textbf{Nhược điểm: }Dung lượng ứng dụng lớn hơn do phải bao gồm Flutter engine.

\section{Tốc độ phát triển}
Có các yếu tố như code reusability, hot reload, các prebuilt component và tính
sẵn có của chúng, quyết định tốc độ phát triển.

\subsection{React Native}
\paragraph{}
\textbf{Hot Reload: }Được hỗ trợ, cho phép các vòng lặp nhanh chóng.
\paragraph{}
\textbf{Code Reusability: }Cao, vì mã JavaScript có thể chạy trên cả iOS và Android.
\paragraph{}
\textbf{Ecosystem: }Hỗ trợ thư viện rộng lớn thông qua npm.
\paragraph{}
Nhưng một trong những tính năng chính nổi tiếng của React Native là khả năng
khởi động phát triển nhanh, bao gồm hot reload và code reusability. Kết hợp
với tất cả các tính năng này và một ecosystem rộng lớn mà nó hỗ trợ, việc
phát triển ứng dụng di động đa nền tảng thường là một lựa chọn hàng đầu.
\paragraph{}
\textbf{Hot Reload:}
\paragraph{}
Hot reload là một trong những tính năng phổ biến nhất
của React Native vì đội ngũ phát triển có thể thấy các thay đổi theo
thời gian thực mà không cần khởi động lại ứng dụng. Nó giúp cắt giảm đáng kể
thời gian của quy trình phát triển vì các lập trình viên có thể nhanh chóng thấy
kết quả của việc sửa đổi code ngay lập tức. Trong trường hợp của một nút bấm
hoặc những thay đổi về hành vi của một component, các thay đổi được áp dụng
trên trình giả lập hoặc thiết bị vật lý ngay lập tức. Bằng cách thực hiện quy trình
phát triển lặp lại, React Native giúp giảm thiểu thời gian chết và mở rộng năng
suất, biến nó thành một lựa chọn khả thi cho các team đang làm việc với
deadline gấp rút.
\paragraph{}
\textbf{Code Reusability:}
\paragraph{}
Với React Native, tái sử dụng mã nguồn là con át chủ bài trong phát triển ứng dụng vì
chúng cho phép các lập trình viên làm việc trên một codebase ứng dụng duy nhất
hoạt động trên cả nền tảng iOS và Android. Điều này được thực hiện bằng
cách sử dụng JavaScript và React framework, mang lại sự tương đồng cho các
chi tiết đặc thù của nền tảng khác nhau thông qua một API thống nhất. Các
component như nút bấm hoặc danh sách có thể được viết một lần và tái sử
dụng trên các nền tảng khác nhau mà không cần thay đổi nhiều. Điều này
không chỉ giảm thời gian phát triển mà còn giảm chi phí bảo trì cũng như
các bản cập nhật chỉ cần áp dụng một lần. Tuy nhiên, các lập trình viên cần viết
native module bằng Java, Objective-C hoặc Swift, điều này có thể mang lại
sự phức tạp.
\paragraph{}
\textbf{Ecosystem:}
\paragraph{}
Một lợi ích thực sự của React Native là nó hưởng lợi từ ecosystem
JavaScript rộng lớn – về cơ bản thông qua npm (Node Package Manager),
cung cấp cho bạn quyền truy cập vào hàng ngàn thư viện và công cụ.
Đây cũng là một ecosystem rộng lớn cho phép các lập trình viên sử dụng
các giải pháp được xây dựng sẵn cho các tác vụ phổ biến, trong đó có
quản lý trạng thái (Redux hoặc MobX), điều hướng (React Navigation)
và tích hợp API (Axios). Hơn nữa, việc tùy chỉnh React Native khá
dễ dàng, và cộng đồng xung quanh React Native cũng lớn và tích cực,
hỗ trợ tài liệu phong phú, hướng dẫn và nhiều tài nguyên bên ngoài.
Điều này giúp các lập trình viên dễ dàng tìm ra giải pháp cho các vấn đề
cũng như các phương pháp tốt nhất.
\begin{minted}[frame=lines, framesep=2mm, baselinestretch=1.2, fontsize=\footnotesize, linenos]{javascript}
import React, { useEffect, useState } from 'react';
import { View, Text, FlatList } from 'react-native';

const App = () => {
  const [data, setData] = useState([]);

  useEffect(() => {
    fetch('https://jsonplaceholder.typicode.com/posts')
      .then(response => response.json())
      .then(json => setData(json))
      .catch(error => console.error(error));
  }, []);
}
\end{minted}
\paragraph{}
\subsection{Kotlin Multiplatform}
\paragraph{}
\textbf{Hot Reload: }Mặc dù các công cụ như KMM (Kotlin Multiplatform Mobile) có
tốc độ phát triển nhanh hơn, nhưng nó không được hỗ trợ natively.
\paragraph{}
\textbf{Code Reusability: }Nó có các Logic nghiệp vụ dùng chung và UI code mang tính
đặc thù nền tảng.
\paragraph{}
\textbf{Ecosystem: }Đang phát triển, nhưng chưa hoàn thiện như React Native hay Flutter.
\paragraph{}
Kotlin Multiplatform (KMM) là một framework hiện đại được thiết kế để cung
cấp mã nguồn hiện đại dùng chung trên nhiều nền tảng mà không cần phải tái cấu trúc mã nguồn
chút nào; Nó cung cấp khả năng tái cấu trúc tự nhiên cho mã đặc thù nền tảng. Phương pháp này
mang lại lợi ích win-win vì nó có sự cân bằng phù hợp giữa khả năng tái sử dụng và tối ưu hóa
theo nền tảng, là một ứng viên sáng giá cho các project cần shared functionality nhưng lại cần
hiệu năng gốc và giao diện mang lại cảm giác quen thuộc cho người dùng.
\paragraph{}
\textbf{Hot Reload:}
\paragraph{}
Kotlin Multiplatform không hỗ trợ hot reload một cách tự nhiên, điều này
có nghĩa là các thay đổi chỉ có hiệu lực khi chạy ứng dụng 
(khác với React Native hay Flutter) và điều này có khả năng 
làm chậm quy trình phát triển, đặc biệt là khi làm việc với các UI 
component. Tuy nhiên, bằng cách sử dụng các công cụ như
Kotlin Multiplatform Mobile (KMM) và sự tích hợp với Android Studio,
việc tăng tốc độ phát triển trở nên dễ dàng hơn nhờ thời gian biên dịch ngắn hơn
và các công cụ gỡ lỗi tốt hơn. Mặc dù các công cụ này giúp giải quyết
một số khó khăn, nhưng tính năng hot reload thực thụ vẫn không có, và các
lập trình viên có thể phải tốn nhiều thời gian hơn để khởi động lại ứng dụng
nhằm xem các thay đổi, điều này kém hiệu quả hơn so với các framework
như Flutter hay React Native.